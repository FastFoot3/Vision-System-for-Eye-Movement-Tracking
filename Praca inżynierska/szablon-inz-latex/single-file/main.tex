% !TeX spellcheck = pl_PL
%%%%%%%%%%%%%%%%%%%%%%%%%%%%%%%%%%%%%%%%%%%
%                                        %
% Szablon pracy dyplomowej inzynierskiej %
% zgodny  z aktualnymi  przepisami  SZJK %
%                                        %
%%%%%%%%%%%%%%%%%%%%%%%%%%%%%%%%%%%%%%%%%%
%                                        %
%  (c) Krzysztof Simiński, 2018-2023     %
%                                        %
%%%%%%%%%%%%%%%%%%%%%%%%%%%%%%%%%%%%%%%%%%
%                                        %
% Najnowsza wersja szablonów jest        %
% podstępna pod adresem                  %
% github.com/ksiminski/polsl-aei-theses  %
%                                        %
%%%%%%%%%%%%%%%%%%%%%%%%%%%%%%%%%%%%%%%%%%
%
%
% Projekt LaTeXowy zapewnia odpowiednie formatowanie pracy,
% zgodnie z wymaganiami Systemu zapewniania jakości kształcenia.
% Proszę nie zmieniać ustawień formatowania (np. fontu,
% marginesów, wytłuszczeń, kursywy itd. ).
%
% Projekt można kompilować na kilka sposobów.
%
% 1. kompilacja pdfLaTeX
%
% pdflatex main
% bibtex   main
% pdflatex main
% pdflatex main
%
%
% 2. kompilacja XeLaTeX
%
% Kompilatacja przy użyciu XeLaTeXa różni się tym, że na stronie
% tytułowej używany jest font Calibri. Wymaga to jego uprzedniego
% zainstalowania.
%
% xelatex main
% bibtex  main
% xelatex main
% xelatex main
%
%
%%%%%%%%%%%%%%%%%%%%%%%%%%%%%%%%%%%%%%%%%%%%%%%%%%%%%
% W przypadku pytań, uwag, proszę pisać na adres:   %
%      krzysztof.siminski(małpa)polsl.pl            %
%%%%%%%%%%%%%%%%%%%%%%%%%%%%%%%%%%%%%%%%%%%%%%%%%%%%%
%
% Chcemy ulepszać szablony LaTeXowe prac dyplomowych.
% Wypełniając ankietę spod poniższego adresu pomogą
% Państwo nam to zrobić. Ankieta jest całkowicie
% anonimowa. Dziękujemy!


% https://docs.google.com/forms/d/e/1FAIpQLScyllVxNKzKFHfILDfdbwC-jvT8YL0RSTFs-s27UGw9CKn-fQ/viewform?usp=sf_link
%
%%%%%%%%%%%%%%%%%%%%%%%%%%%%%%%%%%%%%%%%%%%%%%%%%%%%%%%%%%%%%%%%%%%%%%%%%

%%%%%%%%%%%%%%%%%%%%%%%%%%%%%%%%%%%%%%%%%%%%%%%
%                                             %
% PERSONALIZACJA PRACY – DANE PRACY           %
%                                             %
%%%%%%%%%%%%%%%%%%%%%%%%%%%%%%%%%%%%%%%%%%%%%%%

% Proszę wpisać swoje dane w poniższych definicjach.

%--------------------------------------
%
%
% TODO: None
%
% Status: 100% (DONE)
%--------------------------------------
% dane autora
\newcommand{\FirstNameAuthor}{Bartosz}
\newcommand{\SurnameAuthor}{Wuwer}
\newcommand{\IdAuthor}{296949}   % numer albumu  (bez $\langle$ i $\rangle$)

% drugi autor:
%\newcommand{\FirstNameCoauthor}{Imię}   % Jeżeli jest drugi autor, to tutaj należy podać imię.
%\newcommand{\SurnameCoauthor}{Nazwisko} % Jeżeli jest drugi autor, to tutaj należy podać nazwisko.
%\newcommand{\IdCoauthor}{$\langle$wpisać właściwy$\rangle$}  % numer albumu drugiego autora (bez $\langle$ i $\rangle$)
% Gdy nie ma drugiego autora, należy zostawić poniższe definicje puste, jak poniżej. Gdy jest drugi autor, należy zakomentować te linie.
\newcommand{\FirstNameCoauthor}{} % Jeżeli praca ma tylko jednego autora, to dane drugiego autora zostają puste.
\newcommand{\SurnameCoauthor}{}   % Jeżeli praca ma tylko jednego autora, to dane drugiego autora zostają puste.
\newcommand{\IdCoauthor}{}  % Jeżeli praca ma tylko jednego autora, to dane drugiego autora zostają puste.
%%%%%%%%%%

\newcommand{\Supervisor}{Dr inż. Krzysztof Jaskot}     % dane promotora (bez $\langle$ i $\rangle$)
\newcommand{\Title}{System wizyjny do śledzenia ruchu gałek ocznych}           % tytuł pracy po polsku
\newcommand{\TitleAlt}{Vision system for tracking the movement of the eyeballs}                     % thesis title in English
\newcommand{\Program}{Automatyka i Robotyka}            % kierunek studiów  (bez $\langle$ i $\rangle$)
\newcommand{\Specialisation}{Technologie informacyjne w automatyce i robotyce}     % specjalność  (bez $\langle$ i $\rangle$)
\newcommand{\Departament}{Automatyki i Robotyki}        % katedra promotora  (bez $\langle$ i $\rangle$)

% Jeżeli został wyznaczony promotor pomocniczy lub opiekun, proszę go/ją wpisać ...
\newcommand{\Consultant}{} % dane promotora pomocniczego, opiekuna (bez $\langle$ i $\rangle$)
% ... w przeciwnym razie proszę zostawić puste miejsce jak poniżej:
%\newcommand{\Consultant}{} % brak promotowa pomocniczego / opiekuna

% koniec fragmentu do modyfikacji
%%%%%%%%%%%%%%%%%%%%%%%%%%%%%%%%%%%%%%%%%%


%%%%%%%%%%%%%%%%%%%%%%%%%%%%%%%%%%%%%%%%%%%%%%%
%                                             %
% KONIEC PERSONALIZACJI PRACY                 %
%                                             %
%%%%%%%%%%%%%%%%%%%%%%%%%%%%%%%%%%%%%%%%%%%%%%%

%%%%%%%%%%%%%%%%%%%%%%%%%%%%%%%%%%%%%%%%


%%%%%%%%%%%%%%%%%%%%%%%%%%%%%%%%%%%%%%%%%%%%%%%
%                                             %
% PROSZĘ NIE MODYFIKOWAĆ PONIŻSZYCH USTAWIEŃ! %
%                                             %
%%%%%%%%%%%%%%%%%%%%%%%%%%%%%%%%%%%%%%%%%%%%%%%



\documentclass[a4paper,twoside,12pt]{book}
\usepackage[utf8]{inputenc}                                      
\usepackage[T1]{fontenc}  
\usepackage{amsmath,amsfonts,amssymb,amsthm}
\usepackage[british,polish]{babel} 
\usepackage{indentfirst}
\usepackage{xurl}
\usepackage{xstring}
\usepackage{ifthen}



\usepackage{ifxetex}

\ifxetex
	\usepackage{fontspec}
	\defaultfontfeatures{Mapping=tex—text} % to support TeX conventions like ``——-''
	\usepackage{xunicode} % Unicode support for LaTeX character names (accents, European chars, etc)
	\usepackage{xltxtra} % Extra customizations for XeLaTeX
\else
	\usepackage{lmodern}
\fi



\usepackage[margin=2.5cm]{geometry}
\usepackage{graphicx} 
\usepackage{hyperref}
\usepackage{booktabs}
\usepackage{tikz}
\usepackage{pgfplots}
\usepackage{mathtools}
\usepackage{geometry}
\usepackage{subcaption}   % subfigures
\usepackage[page]{appendix} % toc,
\renewcommand{\appendixtocname}{Dodatki}
\renewcommand{\appendixpagename}{Dodatki}
\renewcommand{\appendixname}{Dodatek}

\usepackage{csquotes}
\usepackage[natbib=true,backend=bibtex,maxbibnames=99]{biblatex}  % kompilacja bibliografii BibTeXem
%\usepackage[natbib=true,backend=biber,maxbibnames=99]{biblatex}  % kompilacja bibliografii Biberem
\bibliography{biblio}

\usepackage{ifmtarg}   % empty commands  

\usepackage{setspace}
\onehalfspacing


\frenchspacing



%%%% TODO LIST GENERATOR %%%%%%%%%

\usepackage{color}
\definecolor{brickred}      {cmyk}{0   , 0.89, 0.94, 0.28}

\makeatletter \newcommand \kslistofremarks{\section*{Uwagi} \@starttoc{rks}}
  \newcommand\l@uwagas[2]
    {\par\noindent \textbf{#2:} %\parbox{10cm}
{#1}\par} \makeatother


\newcommand{\ksremark}[1]{%
{%\marginpar{\textdbend}
{\color{brickred}{[#1]}}}%
\addcontentsline{rks}{uwagas}{\protect{#1}}%
}

\newcommand{\comma}{\ksremark{przecinek}}
\newcommand{\nocomma}{\ksremark{bez przecinka}}
\newcommand{\styl}{\ksremark{styl}}
\newcommand{\ortografia}{\ksremark{ortografia}}
\newcommand{\fleksja}{\ksremark{fleksja}}
\newcommand{\pauza}{\ksremark{pauza `--', nie dywiz `-'}}
\newcommand{\kolokwializm}{\ksremark{kolokwializm}}
\newcommand{\cudzyslowy}{\ksremark{,,polskie cudzysłowy''}}

%%%%%%%%%%%%%% END OF TODO LIST GENERATOR %%%%%%%%%%%

\newcommand{\printCoauthor}{%		
    \StrLen{\FirstNameCoauthor}[\FNCoALen]
    \ifthenelse{\FNCoALen > 0}%
    {%
		{\large\bfseries\Coauthor\par}
	
		{\normalsize\bfseries \LeftId: \IdCoauthor\par}
    }%
    {}
} 

%%%%%%%%%%%%%%%%%%%%%
\newcommand{\autor}{%		
    \StrLen{\FirstNameCoauthor}[\FNCoALenXX]
    \ifthenelse{\FNCoALenXX > 0}%
    {\FirstNameAuthor\ \SurnameAuthor, \FirstNameCoauthor\ \SurnameCoauthor}%
	{\FirstNameAuthor\ \SurnameAuthor}%
}
%%%%%%%%%%%%%%%%%%%%%

\StrLen{\FirstNameCoauthor}[\FNCoALen]
\ifthenelse{\FNCoALen > 0}%
{%
\author{\FirstNameAuthor\ \SurnameAuthor, \FirstNameCoauthor\ \SurnameCoauthor}
}%
{%
\author{\FirstNameAuthor\ \SurnameAuthor}
}%

%%%%%%%%%%%% ZYWA PAGINA %%%%%%%%%%%%%%%
% brak kapitalizacji zywej paginy
\usepackage{fancyhdr}
\pagestyle{fancy}
\fancyhf{}
\fancyhead[LO]{\nouppercase{\it\rightmark}}
\fancyhead[RE]{\nouppercase{\it\leftmark}}
\fancyhead[LE,RO]{\it\thepage}


\fancypagestyle{tylkoNumeryStron}{%
   \fancyhf{} 
   \fancyhead[LE,RO]{\it\thepage}
}

\fancypagestyle{bezNumeracji}{%
   \fancyhf{} 
   \fancyhead[LE,RO]{}
}


\fancypagestyle{NumeryStronNazwyRozdzialow}{%
   \fancyhf{} 
   \fancyhead[LE]{\nouppercase{\autor}}
   \fancyhead[RO]{\nouppercase{\leftmark}} 
   \fancyfoot[CE, CO]{\thepage}
}


%%%%%%%%%%%%% OBCE WTRETY  
\newcommand{\obcy}[1]{\emph{#1}}
\newcommand{\english}[1]{{\selectlanguage{british}\obcy{#1}}}
%%%%%%%%%%%%%%%%%%%%%%%%%%%%%

% polskie oznaczenia funkcji matematycznych
\renewcommand{\tan}{\operatorname {tg}}
\renewcommand{\log}{\operatorname {lg}}

% jeszcze jakies drobiazgi

\newcounter{stronyPozaNumeracja}

%%%%%%%%%%%%%%%%%%%%%%%%%%% 
\newcommand{\printOpiekun}[1]{%		

    \StrLen{\Consultant}[\mystringlen]
    \ifthenelse{\mystringlen > 0}%
    {%
       {\large{\bfseries OPIEKUN, PROMOTOR POMOCNICZY}\par}
       
       {\large{\bfseries \Consultant}\par}
    }%
    {}
} 
%
%%%%%%%%%%%%%%%%%%%%%%%%%%%%%%%%%%%%%%%%%%%%%%
 
% Proszę nie modyfikować poniższych definicji!
\newcommand{\Author}{\FirstNameAuthor\ \MakeUppercase{\SurnameAuthor}} 
\newcommand{\Coauthor}{\FirstNameCoauthor\ \MakeUppercase{\SurnameCoauthor}}
\newcommand{\Type}{PROJEKT INŻYNIERSKI}
\newcommand{\Faculty}{Wydział Automatyki, Elektroniki i Informatyki} 
\newcommand{\Polsl}{Politechnika Śląska}
\newcommand{\Logo}{politechnika_sl_logo_bw_pion_pl.pdf}
\newcommand{\LeftId}{Nr albumu}
\newcommand{\LeftProgram}{Kierunek}
\newcommand{\LeftSpecialisation}{Specjalność}
\newcommand{\LeftSUPERVISOR}{PROWADZĄCY PRACĘ}
\newcommand{\LeftDEPARTMENT}{KATEDRA}
%%%%%%%%%%%%%%%%%%%%%%%%%%%%%%%%%%%%%%%%%%%%%%

%%%%%%%%%%%%%%%%%%%%%%%%%%%%%%%%%%%%%%%%%%%%%%%
%                                             %
% KONIEC USTAWIEŃ                             %
%                                             %
%%%%%%%%%%%%%%%%%%%%%%%%%%%%%%%%%%%%%%%%%%%%%%%




%%%%%%%%%%%%%%%%%%%%%%%%%%%%%%%%%%%%%%%%%%%%%%%
%                                             %
% MOJE PAKIETY, USTAWIENIA ITD                %
%                                             %
%%%%%%%%%%%%%%%%%%%%%%%%%%%%%%%%%%%%%%%%%%%%%%%

% Tutaj proszę umieszczać swoje pakiety, makra, ustawienia itd.


 
%%%%%%%%%%%%%%%%%%%%%%%%%%%%%%%%%%%%%%%%%%%%%%%%%%%%%%%%%%%%%%%%%%%%%
% listingi i fragmentu kodu źródłowego 
% pakiet: listings lub minted
% % % % % % % % % % % % % % % % % % % % % % % % % % % % % % % % % % % 

% biblioteka listings
\usepackage{listings}
\lstset{%
morekeywords={string,exception,std,vector},% słowa kluczowe rozpoznawane przez pakiet listings
language=Python,% C, Matlab, Python, SQL, TeX, XML, bash, ... – vide https://www.ctan.org/pkg/listings
commentstyle=\textit,%
identifierstyle=\textsf,%
keywordstyle=\sffamily\bfseries, %\texttt, %
%captionpos=b,%
tabsize=3,%
frame=lines,%
numbers=left,%
numberstyle=\tiny,%
numbersep=5pt,%
breaklines=true,%
escapeinside={@*}{*@},%
}

% % % % % % % % % % % % % % % % % % % % % % % % % % % % % % % % % % % 
% pakiet minted
%\usepackage{minted}

% pakiet wymaga specjalnego kompilowania:
% pdflatex -shell-escape main.tex
% xelatex  -shell-escape main.tex

%\usepackage[chapter]{minted} % [section]
%%\usemintedstyle{bw}   % czarno-białe kody 
%
%\setminted % https://ctan.org/pkg/minted
%{
%%fontsize=\normalsize,%\footnotesize,
%%captionpos=b,%
%tabsize=3,%
%frame=lines,%
%framesep=2mm,
%numbers=left,%
%numbersep=5pt,%
%breaklines=true,%
%escapeinside=@@,%
%}

%%%%%%%%%%%%%%%%%%%%%%%%%%%%%%%%%%%%%%%%%%%%%%%%%%%%%%%%%%%%%%%%%%%%%



%%%%%%%%%%%%%%%%%%%%%%%%%%%%%%%%%%%%%%%%%%%%%%%
%                                             %
% KONIEC MOICH USTAWIEŃ                       %
%                                             %
%%%%%%%%%%%%%%%%%%%%%%%%%%%%%%%%%%%%%%%%%%%%%%%



%%%%%%%%%%%%%%%%%%%%%%%%%%%%%%%%%%%%%%%%


\begin{document}
%\kslistofremarks

\frontmatter

%%%%%%%%%%%%%%%%%%%%%%%%%%%%%%%%%%%%%%%%%%%%%%%
%                                             %
% PROSZĘ NIE MODYFIKOWAĆ STRONY TYTUŁOWEJ!    %
%                                             %
%%%%%%%%%%%%%%%%%%%%%%%%%%%%%%%%%%%%%%%%%%%%%%%


%%%%%%%%%%%%%%%%%%  STRONA TYTUŁOWA %%%%%%%%%%%%%%%%%%%
\pagestyle{empty}
{
	\newgeometry{top=1.5cm,%
	             bottom=2.5cm,%
	             left=3cm,
	             right=2.5cm}
 
	\ifxetex 
	  \begingroup
	  \setsansfont{Calibri}
	   
	\fi 
	 \sffamily
	\begin{center}
	\includegraphics[width=50mm]{\Logo}
	 
	
	{\Large\bfseries\Type\par}
	
	\vfill  \vfill  
			 
	{\large\Title\par}
	
	\vfill  
		
	{\large\bfseries\Author\par}
	
	{\normalsize\bfseries \LeftId: \IdAuthor}

	\printCoauthor
	
	\vfill  		
 
	{\large{\bfseries \LeftProgram:} \Program\par} 
	
	{\large{\bfseries \LeftSpecialisation:} \Specialisation\par} 
	 		
	\vfill  \vfill 	\vfill 	\vfill 	\vfill 	\vfill 	\vfill  
	 
	{\large{\bfseries \LeftSUPERVISOR}\par}
	
	{\large{\bfseries \Supervisor}\par}
				
	{\large{\bfseries \LeftDEPARTMENT\ \Departament} \par}
		
	{\large{\bfseries \Faculty}\par}
		
	\vfill  \vfill  

    	
    \printOpiekun{\Consultant}
    
	\vfill  \vfill  
		
    {\large\bfseries  Gliwice \the\year}

   \end{center}	
       \ifxetex 
       	  \endgroup
       \fi
	\restoregeometry
}
  
%%%%%%%%%%%%%%%%%%%%%%%%%%%%%%%%%%%%%%%%%%%%%%%
%                                             %
% KONIEC STRONY TYTUŁOWEJ                     %
%                                             %
%%%%%%%%%%%%%%%%%%%%%%%%%%%%%%%%%%%%%%%%%%%%%%%  


\cleardoublepage

\rmfamily\normalfont
\pagestyle{empty}


%%% No to zaczynamy pisać pracę :-) %%%%

%--------------------------------------
%
%
% TODO: Streszczenie, słowa kluczowe, abstract, keywords
%
% Status: 0%
%--------------------------------------
\subsubsection*{Tytuł pracy} 
\Title

\subsubsection*{Streszczenie}  
(Streszczenie pracy – odpowiednie pole w systemie APD powinno zawierać kopię tego streszczenia.)

\subsubsection*{Słowa kluczowe} 
(2-5 slow (fraz) kluczowych, oddzielonych przecinkami)

\subsubsection*{Thesis title} 
\begin{otherlanguage}{british}
\TitleAlt
\end{otherlanguage}

\subsubsection*{Abstract} 
\begin{otherlanguage}{british}
(Thesis abstract – to be copied into an appropriate field during an electronic submission – in English.)
\end{otherlanguage}
\subsubsection*{Key words}  
\begin{otherlanguage}{british}
(2-5 keywords, separated by commas)
\end{otherlanguage}




%%%%%%%%%%%%%%%%%% SPIS TRESCI %%%%%%%%%%%%%%%%%%%%%%
% Add \thispagestyle{empty} to the toc file (main.toc), because \pagestyle{empty} doesn't work if the TOC has multiple pages
\addtocontents{toc}{\protect\thispagestyle{empty}}
\tableofcontents

%%%%%%%%%%%%%%%%%%%%%%%%%%%%%%%%%%%%%%%%%%%%%%%%%%%%%
\setcounter{stronyPozaNumeracja}{\value{page}}
\mainmatter
\pagestyle{empty}

\cleardoublepage

\pagestyle{NumeryStronNazwyRozdzialow}

%%%%%%%%%%%%%% wlasciwa tresc pracy %%%%%%%%%%%%%%%%%

%--------------------------------------
%
%
% TODO: cel pracy (99.9%), zakres pracy(99%), zwięzła charakterystyka rozdziałów
%
% Status: 80%
%--------------------------------------
\chapter{Wstęp}
\label{ch:wstep}

\section{Wprowadzenie i osadzenie w dziedzinie}
% wprowadzenie w problem/zagadnienie
Śledzenie ruchu gałek ocznych (ang. eye-tracking), zwane również jako okulografia, jest techniką która badana jest od ponad stu lat. Istotność tej techniki wynika z faktu, że ruchy gałek ocznych są ściśle związane z procesami poznawczymi, takimi jak uwaga, percepcja, pamięć, czy procesy decyzyjne. Skupiając wzrok na danym punkcie, umieszczamy go w centralnym obszarze naszego pola widzenia które charakteryzuje się największą rozdzielczością, co pozwala na dokładne analizowanie szczegółów. Ten fakt wpływa także na proces skupienia -- gdy koncentrujemy się na danym obiekcie lub obszarze skupiamy na nim wzrok (często wystarczy jedynie krótki moment).

Możliwość rejestrowania ruchów oczu pozwala na zrozumienie w jaki sposób obserwator eksploruje otaczający go świat. Posiadając tą wiedzę możliwe jest wyciągnięcie wniosków na temat tego co jest interesujące lub istotne dla obserwatora, jakie emocje się z tym wiążą, czy nawet jakie procesy poznawcze zachodzą w jego umyśle, czy rozumie on to co widzi. Nie trudnym jest zauważyć jak cenne mogą być te informacje w szerokim spektrum dziedzin.

% osadzenie problemu w dziedzinie
Okulografia odgrywa kluczową rolę w psychologi poznawczej, psychologi społecznej, neurobiologii, marketingu, czy medycynie. W psychologii poznawczej ruch oczu jest ściśle związany z pamięcią, podejmowaniem decyzji, obciążeniem poznawczym i uczeniem się asocjacyjnym. W psychologii społecznej eye-tracking pozwala na wgląd w zachowania społeczne i ich analizę, co pozwala badać empatię, prospołeczność, czy fobie społeczne \cite{bib:tobii-main}. W neurobiologii bada się powiązania ruchu oczu ze szlakami neuronowymi odpowiedzialnymi za podejmowane akcje i procesy myślowe, dając możliwość w diagnozach i wsparciu osób dotkniętych chorobą Parkinsona \cite{bib:tobii-parkinson}, Alzheimera \cite{bib:tobii-alzheimer}, a także autyzmem czy łagodnym upośledzeniem funkcji poznawczych \cite{bib:tobii-autyzm}. W medycynie okulografia pozwala na diagnozę między innymi oczopląsu, który może być objawem np. stwardnienia rozsianego, uszkodzenia śródmózgowia lub urazu ucha wewnętrznego \cite{bib:Diagnostyka-oczoplas}. Niezaprzeczalnie technika ta jest niezwykle cenna i efektywna na wielu płaszczyznach naukowych i praktycznych.

\section{Cel oraz zakres pracy}
% cel pracy
Celem niniejszej pracy jest stworzenie narzędzia które pozwoli na uniwersalne śledzenie ruchu gałek ocznych, bez zdefiniowanej docelowej grupy użytkowników. System ten powinien być prosty w obsłudze i niewymagający zaawansowanej technologii, ale nie ograniczający w razie potrzeby bardziej zaawansowanych użytkowników, prezentując użyteczność zarówno dla naukowców, studentów, jak i hobbystów. Tego typu narzędzie pozwoli na eksplorację danych eye-trackingowych, otwierając możliwości na dalszą analizę i interpretację wyników, a także zapewni solidną podstawę w razie potrzeby modyfikacji lub rozbudowy systemu w bardziej ukierunkowany sposób.

% zakres pracy
Praca ta skupia się przede wszystkim na samym procesie wykrywania źrenic i śledzenia ruchu gałek ocznych na ich podstawie oraz implementacji systemu wizyjnego, który pozwoli na zapis, wizualizację i potencjalną dalszą analizę danych zebranych przez kamerę, w tym także kamerę internetową. Całość zrealizowana jest w języku Python, z wykorzystaniem bibliotek takich jak OpenCV, NumPy, Pandas oraz Matplotlib, a wszelkie prezentowane dane zostały uchwycone przy użyciu wbudowanej kamery laptopa. Praca nie będzie obejmować rozległej analizy zebranych danych, ani eksperymentów przeprowadzonych z użyciem owego systemu. Nie została także stworzona grupa testowa, więc większość testów przeprowadzone zostały na autorze pracy.

\begin{itemize}
\item wprowadzenie w problem/zagadnienie
\item osadzenie problemu w dziedzinie
\item cel pracy
\item zakres pracy
\item zwięzła charakterystyka rozdziałów
\item jednoznaczne określenie wkładu autora, w przypadku prac wieloosobowych – tabela z autorstwem poszczególnych elementów pracy
\end{itemize}



%--------------------------------------
%
%
% TODO: sformułowanie problemu [płynne podążanie, fiksacja, wizualizacja modeli] (75% ), state of the art, studia literaturowe
%
% Status: 32%
%--------------------------------------
\chapter{[Analiza tematu]}

% =========== wzorowanie się na przykładzie ===========
Szczegółowe sformułowanie problemu
Śledzenie ruchu gałek ocznych (ang. eye-tracking) to proces rejestrowania i analizy sposobu, w jaki ludzkie oczy poruszają się w trakcie obserwacji otoczenia lub analizowania określonych bodźców wizualnych. Problem ten jest istotny, ponieważ ludzki wzrok jest kluczowym narzędziem percepcji, a sposób, w jaki poruszamy oczami, dostarcza bezpośrednich informacji o mechanizmach uwagi, percepcji i procesów decyzyjnych. Oczy rejestrują jedynie niewielki obszar otoczenia w wysokiej rozdzielczości (tzw. plamka żółta), co wymaga ciągłego ruchu w celu skupienia wzroku na interesujących obszarach.

Główne pytania badawcze dotyczą:

Identyfikacji mechanizmów leżących u podstaw ruchów oczu, takich jak sakady, fiksacje czy dryfy;
Analizy trajektorii ruchów w celu interpretacji procesów poznawczych, takich jak uwaga selektywna czy eksploracja wizualna;
Opracowania efektywnych metod analizy i wizualizacji dużych zbiorów danych generowanych podczas badań eye-trackingowych.
Współczesny rozwój technologii eye-trackingu umożliwił precyzyjną rejestrację trajektorii ruchów gałek ocznych, jednak nadal istnieją wyzwania związane z interpretacją wyników, skalowalnością analiz oraz adaptacją narzędzi do różnych dziedzin badawczych i aplikacji.

Osadzenie tematu w kontekście aktualnego stanu wiedzy (state of the art)
Eye-tracking to technologia zyskująca na popularności dzięki postępowi w precyzji urządzeń rejestrujących oraz spadkowi ich kosztów. Obecny stan wiedzy obejmuje zarówno aspekty techniczne, jak i teoretyczne związane z analizą danych eye-trackingowych.

Technologie i urządzenia eye-trackingowe:
Najnowocześniejsze urządzenia korzystają z technologii optycznych i wideo, takich jak systemy montowane na głowie (np. okulary rejestrujące ruchy oczu) oraz stacjonarne urządzenia rejestrujące pozycję oka w czasie rzeczywistym. Firmy takie jak Tobii, EyeLink czy Pupil Labs dominują na rynku, oferując rozwiązania różniące się precyzją, mobilnością i kosztem.

Zastosowania eye-trackingu:
W psychologii i neurobiologii eye-tracking wykorzystywany jest do badania procesów uwagi, percepcji oraz podejmowania decyzji. W marketingu umożliwia analizę wzorców oglądania reklam, a w interfejsach człowiek-komputer pozwala na projektowanie bardziej intuicyjnych systemów. W medycynie stosuje się go m.in. do diagnostyki chorób neurologicznych, takich jak autyzm czy choroba Parkinsona.

Ograniczenia technologiczne i wyzwania:
Pomimo rozwoju technologii, istnieją trudności w interpretacji wyników badań. Brak uniwersalnych standardów analizy, a także różnorodność dostępnych algorytmów detekcji fiksacji i sakad, sprawiają, że analiza ruchów oczu wymaga znacznego doświadczenia i dostosowywania parametrów.

Studia literaturowe – opis znanych rozwiązań
Badania naukowe i literatura fachowa dostarczają licznych przykładów metod i algorytmów stosowanych w analizie danych eye-trackingowych. Poniżej przedstawiono najważniejsze rozwiązania opisane w literaturze.

Algorytmy analizy trajektorii ruchów oczu:

Algorytmy detekcji fiksacji:
Velocity-Threshold Identification (I-VT) – metoda oparta na progach prędkości ruchu gałki ocznej, używana do identyfikacji okresów stabilnego spojrzenia (fiksacji) oraz szybkich ruchów (sakad).
Dispersion-Threshold Identification (I-DT) – bazuje na obliczaniu dyspersji (rozproszenia) punktów spojrzenia w określonym przedziale czasowym.
Algorytmy segmentacji trajektorii:
Hidden Markov Models (HMM) – modele statystyczne używane do identyfikacji sekwencji ruchów oczu w danych złożonych.
Clustering-Based Methods – wykorzystujące algorytmy grupowania, takie jak k-średnie, w celu klasyfikacji punktów spojrzenia.
Przykłady narzędzi i oprogramowania:

Tobii Pro Lab – komercyjne oprogramowanie pozwalające na analizę i wizualizację danych eye-trackingowych, szeroko stosowane w badaniach naukowych i marketingowych.
PsychoPy i PyGaze – otwarte platformy do projektowania eksperymentów i analizy danych.
OpenGaze – inicjatywa open-source rozwijająca algorytmy analizy i wizualizacji danych eye-trackingowych.
Porównanie podejść:
Badania naukowe, takie jak [Duchowski, 2017], podkreślają, że wybór metody analizy zależy od charakteru danych i celów badawczych. Metody proste, takie jak I-VT, są skuteczne w analizie podstawowych wzorców, natomiast bardziej zaawansowane podejścia, jak HMM, oferują większą elastyczność, ale kosztem złożoności obliczeniowej.

Podsumowanie:
Studia literaturowe wskazują, że rozwój algorytmów i narzędzi analizy danych eye-trackingowych jest kluczowy dla efektywnego wykorzystania tej technologii w badaniach naukowych i praktycznych zastosowaniach. Jednak wciąż istnieją wyzwania związane z interpretacją wyników, integracją danych z różnych źródeł oraz tworzeniem intuicyjnych narzędzi do analizy.

% =========== moja część ===========
% sformułowanie problemu
Widzeniem plamkowym (fovealnym) nazywamy wspomniany już wcześniej fakt, że widzimy w największej rozdzielczości jedynie centrum pola widzenia. Wynika to z budowy siatkówki, która posiada plamkę żółtą (fovea), jest to niewielki obszar, który charakteryzuje się największym zagęszczeniem czopków. Taka budowa siatkówki, która odpowiada za odbiór bodźców wzrokowych, sprawia, że nieustannie poruszamy oczami. Skupiając wzrok na danym punkcie, musimy być gotowi na to, że cała reszta pola widzenia straci dla nas na ostrości, jest to proces, który przypomina filtracje wielu otaczających nas informacji, do paru, które mają w tym momencie znaczenie.

Uwaga wzrokowa może zostać opisana przez idiomy "gdzie" i "co", które pozwalają na zobrazowanie jej selektywnej natury. "Gdzie" to proces wizualnego wyszukiwania i wyboru lokalizacji, które zwróciło naszą uwagę, w celu dokładniejszego zbadania. Istotnym aspektem tego procesu selekcji jest widzenie peryferyjne, czyli te oddalone od naszego punktu skupienia, ale wciąż pozwalające na wyodrębnienie kształtów i ruchu, co w pewien sposób prowadzi nasze centralne spojrzenie. Przykładem może być spojrzenie przez okno, w pierwszym momencie nasz wzrok kierowany jest na wyraźne kształty, światła czy nagły ruch taki jak np. lądujący ptak. "Co" można nazwać odwrotnością "gdzie", jest to proces szczegółowego badania danego obszaru, charakteryzują go takie zjawiska jak fiksacja, czyli stabilizacja wzroku na danym punkcie oraz ruchy sakadowe, czyli szybkie mimowolne ruchy oczu pomiędzy kolejnymi punktami, które pozwalają na obserwacje. Całość tworzy kanał percepcyjny o ograniczonym zasięgu przestrzennym, obejmujący interesujący obszar, budując w ten sposób naszą świadomość i zrozumienie celu naszego spojrzenia. Wracając do poprzedniego przykładu, po zauważeniu ptaka, nasze spojrzenie skupi się na nim (fiksacja), a ruchy sakadowe oczu pozwolą na dokładne zbadanie jego kształtu, koloru i detali. To właśnie połączenie "gdzie" i "co" pozwala na głębsze zrozumienie otaczającego nas świata, a owe ścieżki skanowania dają nam wgląd w badaną osobę, jako że pomiar widzenia fovealnego w czasie odzwierciedla chwilowe i jawne skupienie uwagi wzrokowej obserwatora. 

Tutaj warto wspomnieć o poszukiwaniu wzrokowym, jest to proces aktywnego przeszukiwania pola widzenia w celu znalezienia konkretnego celu, a częścią tego procesu jest etap przeduwagowy. Etap ten równolegle analizuje duży obszar widzenia jednocześnie, dzieje się to automatycznie, nieświadomie i nie wymaga widzenia fovealnego. Co charakteryzuje ten proces to umiejętność rozpoznania czterech podstawowych cech: kolor, rozmiar, orientacja oraz obecność i/lub kierunek ruchu. Etap przeduwagowy jest pierwszym krokiem poszukiwania wzrokowego i obejmuje dużą część poszukiwania typu równoległego (np. próba zauważenia nagłego ruchu spadającej gwiazdy) i relatywnie niewielką część poszukiwania typu seryjnego (np. przeszukiwanie obiektów na stole w poszukiwaniu kluczy), które wymaga uwagi i przenoszenia wzroku od obiektu do obiektu. 

Powyższe rozważania pozwalają przejść do głównego nurtu tematu śledzenia ruchu gałek ocznych, czyli analiza ruchów oczu. Rozróżniamy pięć podstawowych typów ruchów oczu: wspomniane już wcześniej sakadyczne, wergencyjne, przedsionkowe, płynne podążanie (smooth pursuit) oraz oczopląs fizjologiczny, który jest naturalny dla zdrowej osoby i często występuje podczas fiksacji. Ruchy te można podzielić na dobrowolne, mimowolne i odruchowe, a sygnały je kontrolujące pochodzą z obszarów korowych mózgu. Do opisania tych procesów można posłużyć się modelowaniem matematycznym. Do zrozumienia jawnej uwagi wzrokowej wystarczy modelowanie trzech typów ruchów: fiksacji, która pokazuje chęć utrzymania wzroku na stacjonarnym obiekcie, sakad, które mogą wskazywać na chęć zmiany punktu uwagi oraz płynnych pościgów, które podobnie do fiksacji, pozwalają na śledzenie ruchomego obiektu. 

Sygnał sakad można opisać jako funkcja impuls/skok, gdzie impuls na wejściu reprezentuje prędkość, a skok pozycję, impuls jest przepuszczany przez filtr, który przekształca go w skok. Prostą reprezentacją ruchu sakadowego jest filtr liniowy różniczkujący, który dokonuje potrzebnej konwersji informacji prędkość na przemieszczenie. Wzór owego filtra w dziedzinie czasu można zapisać następująco

\begin{align*}
	x_t &= g_0 \cdot s_t + g_1 \cdot s_{t-1} + g_2 \cdot s_{t-2} + \ldots
\end{align*}

Wyjście $x_t$ zależy od bieżącego wejścia $s_t$ oraz jego poprzednich wartości $s_{t-1}, s_{t-2}, \ldots$ ważonych odpowiednimi współczynnikami filtra $(g_0, g_1, g_2, \ldots)$. W zapisie sumarycznym wzór prezentuje się następująco

\begin{align*}
	x_t &= \sum_{k=0}^{\infty} g_k \cdot s_{t-k}
\end{align*}

Filtr Haara jest jednym z przykładów filtrów który przybliża różniczkowanie, jest to filtr o długości 2, czyli operujący jedynie na dwóch kolejnych próbkach sygnału wejściowego $(s_t, s_{t-1})$, a przybliża on pierwszą pochodną, przyjmując, że prędkość zmian sygnału jest stała w czasie trwania dwóch próbek. Współczynniki filtra są równe $g_0 = 1$ oraz $g_1 = -1$, co oznacza że różniczkowanie jest zrealizowane poprzez różnicę między dwiema kolejnymi próbkami sygnału. W związku z tym transmitancję filtru Haara można zapisać jako

\begin{align*}
	x_t &= g_0 \cdot s_t + g_1 \cdot s_{t-1} \\
	x_t &= 1 \cdot s_t - 1 \cdot s_{t-1} \\
	x_t &= s_t - s_{t-1} \\
	X(z) &= (1 - z) \cdot S(z) \\
	\frac{X(z)}{S(z)} &= 1 - z
\end{align*}

Płynne podążanie występuje gdy obserwator śledzi obiekt w ruchu, ruch ten oczywiście nie może być zbyt gwałtowny. Tego typu śledzenie ruchu jest przykładem systemu sterowania z ujemnym sprzężeniem zwrotnym. Do modelowania owych ruchów używana jest prosta pętla, którą można opisać nastepujacym równaniem w dziedzinie czasu

\begin{align*}
	h(s_t - x_t) &= x_{t+1}
\end{align*}

Wejście $s_t$ to pozycja celu, wyjście $x_t$ to pożądana pozycja oka. Można zauważyć, że w tym przypadku nie jest wymagana transformacja informacji wejściowej na wyjściową, a jedynie zmodyfikowanie jej wartości, w związku z tym $h$ jest liniowym niezmiennym w czasie filtrem, czyli po prostu wzmocnieniem systemu. Przeprowadźmy teraz transformację Laurenta na powyższym równaniu by uzyskać transmitancję systemu

\begin{align*}
	\mathcal{Z}\{h(s_t - x_t)\} &= \mathcal{Z}\{x_{t+1}\} \\
	H(z) \cdot (S(z) - X(z)) &=  X(z) \\
	H(z) \cdot S(z) - H(z) \cdot X(z) &=  X(z) \\
	H(z) \cdot S(z) &= (H(z) + 1) \cdot X(z) \\
	\frac{X(z)}{S(z)} &= \frac{H(z)}{H(z) + 1}
\end{align*}

W ten sposób płynne sygnał z receptorów oka służy za błąd, który następnie jest kompensowany w celu utrzymania obrazu w przestrzeni widzenia plamkowego.





Ruch gałek ocznych i związane z nim widzenie jest bardzo szerokim tematem interdyscyplinarnym, na potrzeby tego projektu powyższe sformułowanie problemu, chociaż uproszczone i nie wyczerpujące, pozwala na zrozumienie w pełni istoty prezentowanego systemu.



\begin{itemize}
\item sformułowanie problemu
\item osadzenie tematu w kontekście aktualnego stanu wiedzy (\english{state of the art}) o poruszanym problemie
\item  studia literaturowe \cite{bib:artykul,bib:ksiazka,bib:konferencja,bib:internet} -  opis znanych rozwiązań (także opisanych naukowo, jeżeli problem jest poruszany w publikacjach naukowych), algorytmów, 
\end{itemize}


Wzory  
\begin{align}
y = \frac{\partial x}{\partial t}
\end{align}
jak i pojedyncze symbole $x$ i $y$  składa się w trybie matematycznym.


%%%%%%%%%%%%%%%%%%%%%%%%





%--------------------------------------
%
%
% TODO: wymagania funkcjonalne i niefunkcjonalne, przypadki użycia, opis narzędzi itp., metodyka pracy
%
% Status: 0%
%--------------------------------------
\chapter{Wymagania i narzędzia}
\label{ch:wymagania-i-narzedzia}

\begin{itemize}
\item wymagania funkcjonalne i niefunkcjonalne
\item przypadki użycia (diagramy UML) -- dla prac, w których mają zastosowanie
\item opis narzędzi, metod eksperymentalnych, metod modelowania itp.
\item metodyka pracy nad projektowaniem i implementacją -- dla prac, w których ma to zastosowanie
\end{itemize}


%--------------------------------------
%
%
% TODO: wymagania sprzętowe i programowe, sposób instalacji, aktywacji, kategorie użytkowników, sposób obsługi, administracja systemem, kwestie bezpieczeństwa, przykład działania, scenariusze korzystania z systemu
%
% Status: 0%
%--------------------------------------
\chapter{[Właściwy dla kierunku -- np. Specyfikacja zewnętrzna]}
\label{ch:04}

Jeśli „Specyfikacja zewnętrzna”:
\begin{itemize}
\item  wymagania sprzętowe i programowe
\item  sposób instalacji
\item  sposób aktywacji
\item  kategorie użytkowników
\item  sposób obsługi
\item  administracja systemem
\item  kwestie bezpieczeństwa
\item  przykład działania
\item  scenariusze korzystania z systemu (ilustrowane zrzutami z ekranu lub generowanymi dokumentami)
\end{itemize}

%%%%%%%%%%%%%%%%%%%%%
%% RYSUNEK Z PLIKU
%
%\begin{figure}
%\centering
%\includegraphics[width=0.5\textwidth]{./politechnika_sl_logo_bw_pion_pl.pdf}
%\caption{Podpis rysunku zawsze pod rysunkiem.}
%\label{fig:etykieta-rysunku}
%\end{figure}
%Rys. \ref{fig:etykieta-rysunku} przestawia …
%%%%%%%%%%%%%%%%%%%%%
%
%%%%%%%%%%%%%%%%%%%%%
%% WIELE RYSUNKÓW 
%
%\begin{figure}
%\centering
%\begin{subfigure}{0.4\textwidth}
%    \includegraphics[width=\textwidth]{./politechnika_sl_logo_bw_pion_pl.pdf}
%    \caption{Lewy górny rysunek.}
%    \label{fig:lewy-gorny}
%\end{subfigure}
%\hfill
%\begin{subfigure}{0.4\textwidth}
%    \includegraphics[width=\textwidth]{./politechnika_sl_logo_bw_pion_pl.pdf}
%    \caption{Prawy górny rysunek.}
%    \label{fig:prawy-gorny}
%\end{subfigure}
%
%\begin{subfigure}{0.4\textwidth}
%    \includegraphics[width=\textwidth]{./politechnika_sl_logo_bw_pion_pl.pdf}
%    \caption{Lewy dolny rysunek.}
%    \label{fig:lewy-dolny}
%\end{subfigure}
%\hfill
%\begin{subfigure}{0.4\textwidth}
%    \includegraphics[width=\textwidth]{./politechnika_sl_logo_bw_pion_pl.pdf}
%    \caption{Prawy dolny rysunek.}
%    \label{fig:prawy-dolny}
%\end{subfigure}
%        
%\caption{Wspólny podpis kilku rysunków.}
%\label{fig:wiele-rysunkow}
%\end{figure}
%Rys. \ref{fig:wiele-rysunkow} przestawia wiele ważnych informacji, np. rys. \ref{fig:prawy-gorny} jest na prawo u góry.
%%%%%%%%%%%%%%%%%%%%%


 
\begin{figure}
\centering
\begin{tikzpicture}
\begin{axis}[
    y tick label style={
        /pgf/number format/.cd,
            fixed,   % po zakomentowaniu os rzednych jest indeksowana wykladniczo
            fixed zerofill, % 1.0 zamiast 1
            precision=1,
        /tikz/.cd
    },
    x tick label style={
        /pgf/number format/.cd,
            fixed,
            fixed zerofill,
            precision=2,
        /tikz/.cd
    }
]
\addplot [domain=0.0:0.1] {rnd};
\end{axis} 
\end{tikzpicture}
\caption{Podpis rysunku po rysunkiem.}
\label{fig:2}
\end{figure}



%--------------------------------------
%
%
% TODO: przedstawienie idei, architektura systemu, opis struktur danych, komponenty itp. (jeśli występują), przegląd ważniejszych algorytmów (jeśli występują), szczegóły implementacji wybranych fragmentów, diagramy UML
%
% Status: 0%
%--------------------------------------
\chapter{[Właściwy dla kierunku -- np. Specyfikacja wewnętrzna]}
\label{ch:05}


Jeśli „Specyfikacja wewnętrzna”:
\begin{itemize}
\item przedstawienie idei
\item architektura systemu
\item opis struktur danych (i organizacji baz danych)
\item komponenty, moduły, biblioteki, przegląd ważniejszych klas (jeśli występują)
\item przegląd ważniejszych algorytmów (jeśli występują)
\item szczegóły implementacji wybranych fragmentów, zastosowane wzorce projektowe
\item diagramy UML
\end{itemize}

% % % % % % % % % % % % % % % % % % % % % % % % % % % % % % % % % % % 
% Pakiet minted wymaga importu: \usepackage{minted}                 %
% i specjalnego kompilowania:                                       %
% pdflatex -shell-escape main                                       %
% % % % % % % % % % % % % % % % % % % % % % % % % % % % % % % % % % % 


Krótka wstawka kodu w linii tekstu jest możliwa, np.  \lstinline|int a;| (biblioteka \texttt{listings})% lub  \mintinline{C++}|int a;| (biblioteka \texttt{minted})
. 
Dłuższe fragmenty lepiej jest umieszczać jako rysunek, np. kod na rys \ref{fig:pseudokod:listings}% i rys. \ref{fig:pseudokod:minted}
, a naprawdę długie fragmenty – w załączniku.


\begin{figure}
\centering
\begin{lstlisting}
class test : public basic
{
    public:
      test (int a);
      friend std::ostream operator<<(std::ostream & s, 
                                     const test & t);
    protected:
      int _a;  
      
};
\end{lstlisting}
\caption{Pseudokod w \texttt{listings}.}
\label{fig:pseudokod:listings}
\end{figure}

%\begin{figure}
%\centering
%\begin{minted}[linenos,frame=lines]{c++}
%class test : public basic
%{
%    public:
%      test (int a);
%      friend std::ostream operator<<(std::ostream & s, 
%                                     const test & t);
%    protected:
%      int _a;  
%      
%};
%\end{minted}
%\caption{Pseudokod w \texttt{minted}.}
%\label{fig:pseudokod:minted}
%\end{figure}




%--------------------------------------
%
%
% TODO: sposób testowania w ramach pracy, organizacja eksperymentów, przypadki testowe zakres testowania, wykryte i usunięte błędy, opcjonalnie wyniki badań eksperymentalnych
%
% Status: 0%
%--------------------------------------
\chapter{Weryfikacja i walidacja}
\label{ch:06}
\begin{itemize}
\item sposób testowania w ramach pracy (np. odniesienie do modelu V)
\item organizacja eksperymentów
\item przypadki testowe zakres testowania (pełny/niepełny)
\item wykryte i usunięte błędy
\item opcjonalnie wyniki badań eksperymentalnych
\end{itemize}

\begin{table}
\centering
\caption{Nagłówek tabeli jest nad tabelą.}
\label{id:tab:wyniki}
\begin{tabular}{rrrrrrrr}
\toprule
	         &                                     \multicolumn{7}{c}{metoda}                                      \\
	         \cmidrule{2-8}
	         &         &         &        \multicolumn{3}{c}{alg. 3}        & \multicolumn{2}{c}{alg. 4, $\gamma = 2$} \\
	         \cmidrule(r){4-6}\cmidrule(r){7-8}
	$\zeta$ &     alg. 1 &   alg. 2 & $\alpha= 1.5$ & $\alpha= 2$ & $\alpha= 3$ &   $\beta = 0.1$  &   $\beta = -0.1$ \\
\midrule
	       0 &  8.3250 & 1.45305 &       7.5791 &    14.8517 &    20.0028 & 1.16396 &                       1.1365 \\
	       5 &  0.6111 & 2.27126 &       6.9952 &    13.8560 &    18.6064 & 1.18659 &                       1.1630 \\
	      10 & 11.6126 & 2.69218 &       6.2520 &    12.5202 &    16.8278 & 1.23180 &                       1.2045 \\
	      15 &  0.5665 & 2.95046 &       5.7753 &    11.4588 &    15.4837 & 1.25131 &                       1.2614 \\
	      20 & 15.8728 & 3.07225 &       5.3071 &    10.3935 &    13.8738 & 1.25307 &                       1.2217 \\
	      25 &  0.9791 & 3.19034 &       5.4575 &     9.9533 &    13.0721 & 1.27104 &                       1.2640 \\
	      30 &  2.0228 & 3.27474 &       5.7461 &     9.7164 &    12.2637 & 1.33404 &                       1.3209 \\
	      35 & 13.4210 & 3.36086 &       6.6735 &    10.0442 &    12.0270 & 1.35385 &                       1.3059 \\
	      40 & 13.2226 & 3.36420 &       7.7248 &    10.4495 &    12.0379 & 1.34919 &                       1.2768 \\
	      45 & 12.8445 & 3.47436 &       8.5539 &    10.8552 &    12.2773 & 1.42303 &                       1.4362 \\
	      50 & 12.9245 & 3.58228 &       9.2702 &    11.2183 &    12.3990 & 1.40922 &                       1.3724 \\
\bottomrule
\end{tabular}
\end{table}  



%--------------------------------------
%
%
% TODO: uzyskane wyniki, kierunki ewentualnych dalszych prac, problemy napotkane w trakcie pracy, wnioski
%
% Status: 0%
%--------------------------------------
\chapter{Podsumowanie i wnioski}
\begin{itemize}
\item uzyskane wyniki w świetle postawionych celów i zdefiniowanych wyżej wymagań
\item kierunki ewentualnych danych prac (rozbudowa funkcjonalna …)
\item problemy napotkane w trakcie pracy
\end{itemize}



\backmatter

%\bibliographystyle{plplain}  % bibtex
%\bibliography{biblio} % bibtex
\printbibliography           % biblatex
\addcontentsline{toc}{chapter}{Bibliografia}

\begin{appendices}

%--------------------------------------
%
%
% TODO: skróty z tekstu
%
% Status: ?%
%--------------------------------------
\chapter{Spis skrótów i symboli}

\begin{itemize}
\item[DNA] kwas deoksyrybonukleinowy (ang. \english{deoxyribonucleic acid})
\item[MVC] model -- widok -- kontroler (ang. \english{model--view--controller}) 
\item[$N$] liczebność zbioru danych
\item[$\mu$] stopnień przyleżności do zbioru
\item[$\mathbb{E}$] zbiór krawędzi grafu
\item[$\mathcal{L}$] transformata Laplace'a 
\end{itemize}


%--------------------------------------
%
%
% TODO: długi fragment źródła
%
% Status: ?%
%--------------------------------------
\chapter{Źródła}

Jeżeli w pracy konieczne jest umieszczenie długich fragmentów kodu źródłowego, należy je przenieść w to miejsce.

\begin{lstlisting}
if (_nClusters < 1)
	throw std::string ("unknown number of clusters");
if (_nIterations < 1 and _epsilon < 0)
	throw std::string ("You should set a maximal number of iteration or minimal difference -- epsilon.");
if (_nIterations > 0 and _epsilon > 0)
	throw std::string ("Both number of iterations and minimal epsilon set -- you should set either number of iterations or minimal epsilon.");
\end{lstlisting}


% % % % % % % % % % % % % % % % % % % % % % % % % % % % % % % % % % % 
% Pakiet minted wymaga odkomentowania w pliku config/settings.tex   %
% importu pakietu minted: \usepackage{minted}                       %
% i specjalnego kompilowania:                                       %
% pdflatex -shell-escape praca                                      %
% % % % % % % % % % % % % % % % % % % % % % % % % % % % % % % % % % % 

%\begin{minted}[linenos,breaklines,frame=lines]{c++}
%if (_nClusters < 1)
%   throw std::string ("unknown number of clusters");
%if (_nIterations < 1 and _epsilon < 0)
%   throw std::string ("You should set a maximal number of iteration or minimal difference -- epsilon.");
%if (_nIterations > 0 and _epsilon > 0)
%   throw std::string ("Both number of iterations and minimal epsilon set -- you should set either number of iterations or minimal epsilon.");
%\end{minted}


%--------------------------------------
%
%
% TODO: źródła programu, dane testowe, film pokazujący działanie oprogramowania
%
% Status: 0%
%--------------------------------------
\chapter{Lista dodatkowych plików, uzupełniających tekst pracy} 


W systemie do pracy dołączono dodatkowe pliki zawierające:
\begin{itemize}
\item źródła programu,
\item dane testowe,
\item film pokazujący działanie opracowanego oprogramowania lub zaprojektowanego i~wykonanego urządzenia,
\item itp.
\end{itemize}


\listoffigures
\addcontentsline{toc}{chapter}{Spis rysunków}
\listoftables
\addcontentsline{toc}{chapter}{Spis tabel}

\end{appendices}

\end{document}


%% Finis coronat opus.

