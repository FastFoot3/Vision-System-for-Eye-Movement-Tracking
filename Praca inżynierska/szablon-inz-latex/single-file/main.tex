% !TeX spellcheck = pl_PL
%%%%%%%%%%%%%%%%%%%%%%%%%%%%%%%%%%%%%%%%%%%
%                                        %
% Szablon pracy dyplomowej inzynierskiej %
% zgodny  z aktualnymi  przepisami  SZJK %
%                                        %
%%%%%%%%%%%%%%%%%%%%%%%%%%%%%%%%%%%%%%%%%%
%                                        %
%  (c) Krzysztof Simiński, 2018-2023     %
%                                        %
%%%%%%%%%%%%%%%%%%%%%%%%%%%%%%%%%%%%%%%%%%
%                                        %
% Najnowsza wersja szablonów jest        %
% podstępna pod adresem                  %
% github.com/ksiminski/polsl-aei-theses  %
%                                        %
%%%%%%%%%%%%%%%%%%%%%%%%%%%%%%%%%%%%%%%%%%
%
%
% Projekt LaTeXowy zapewnia odpowiednie formatowanie pracy,
% zgodnie z wymaganiami Systemu zapewniania jakości kształcenia.
% Proszę nie zmieniać ustawień formatowania (np. fontu,
% marginesów, wytłuszczeń, kursywy itd. ).
%
% Projekt można kompilować na kilka sposobów.
%
% 1. kompilacja pdfLaTeX
%
% pdflatex main
% bibtex   main
% pdflatex main
% pdflatex main
%
%
% 2. kompilacja XeLaTeX
%
% Kompilatacja przy użyciu XeLaTeXa różni się tym, że na stronie
% tytułowej używany jest font Calibri. Wymaga to jego uprzedniego
% zainstalowania.
%
% xelatex main
% bibtex  main
% xelatex main
% xelatex main
%
%
%%%%%%%%%%%%%%%%%%%%%%%%%%%%%%%%%%%%%%%%%%%%%%%%%%%%%
% W przypadku pytań, uwag, proszę pisać na adres:   %
%      krzysztof.siminski(małpa)polsl.pl            %
%%%%%%%%%%%%%%%%%%%%%%%%%%%%%%%%%%%%%%%%%%%%%%%%%%%%%
%
% Chcemy ulepszać szablony LaTeXowe prac dyplomowych.
% Wypełniając ankietę spod poniższego adresu pomogą
% Państwo nam to zrobić. Ankieta jest całkowicie
% anonimowa. Dziękujemy!


% https://docs.google.com/forms/d/e/1FAIpQLScyllVxNKzKFHfILDfdbwC-jvT8YL0RSTFs-s27UGw9CKn-fQ/viewform?usp=sf_link
%
%%%%%%%%%%%%%%%%%%%%%%%%%%%%%%%%%%%%%%%%%%%%%%%%%%%%%%%%%%%%%%%%%%%%%%%%%

%%%%%%%%%%%%%%%%%%%%%%%%%%%%%%%%%%%%%%%%%%%%%%%
%                                             %
% PERSONALIZACJA PRACY – DANE PRACY           %
%                                             %
%%%%%%%%%%%%%%%%%%%%%%%%%%%%%%%%%%%%%%%%%%%%%%%

% Proszę wpisać swoje dane w poniższych definicjach.

%--------------------------------------
%
%
% TODO: None
%
% Status: 100% (DONE)
%--------------------------------------
% dane autora
\newcommand{\FirstNameAuthor}{Bartosz}
\newcommand{\SurnameAuthor}{Wuwer}
\newcommand{\IdAuthor}{296949}   % numer albumu  (bez $\langle$ i $\rangle$)

% drugi autor:
%\newcommand{\FirstNameCoauthor}{Imię}   % Jeżeli jest drugi autor, to tutaj należy podać imię.
%\newcommand{\SurnameCoauthor}{Nazwisko} % Jeżeli jest drugi autor, to tutaj należy podać nazwisko.
%\newcommand{\IdCoauthor}{$\langle$wpisać właściwy$\rangle$}  % numer albumu drugiego autora (bez $\langle$ i $\rangle$)
% Gdy nie ma drugiego autora, należy zostawić poniższe definicje puste, jak poniżej. Gdy jest drugi autor, należy zakomentować te linie.
\newcommand{\FirstNameCoauthor}{} % Jeżeli praca ma tylko jednego autora, to dane drugiego autora zostają puste.
\newcommand{\SurnameCoauthor}{}   % Jeżeli praca ma tylko jednego autora, to dane drugiego autora zostają puste.
\newcommand{\IdCoauthor}{}  % Jeżeli praca ma tylko jednego autora, to dane drugiego autora zostają puste.
%%%%%%%%%%

\newcommand{\Supervisor}{Dr inż. Krzysztof Jaskot}     % dane promotora (bez $\langle$ i $\rangle$)
\newcommand{\Title}{System wizyjny do śledzenia ruchu gałek ocznych}           % tytuł pracy po polsku
\newcommand{\TitleAlt}{Vision system for tracking the movement of the eyeballs}                     % thesis title in English
\newcommand{\Program}{Automatyka i Robotyka}            % kierunek studiów  (bez $\langle$ i $\rangle$)
\newcommand{\Specialisation}{Technologie informacyjne w automatyce i robotyce}     % specjalność  (bez $\langle$ i $\rangle$)
\newcommand{\Departament}{Automatyki i Robotyki}        % katedra promotora  (bez $\langle$ i $\rangle$)

% Jeżeli został wyznaczony promotor pomocniczy lub opiekun, proszę go/ją wpisać ...
\newcommand{\Consultant}{} % dane promotora pomocniczego, opiekuna (bez $\langle$ i $\rangle$)
% ... w przeciwnym razie proszę zostawić puste miejsce jak poniżej:
%\newcommand{\Consultant}{} % brak promotowa pomocniczego / opiekuna

% koniec fragmentu do modyfikacji
%%%%%%%%%%%%%%%%%%%%%%%%%%%%%%%%%%%%%%%%%%


%%%%%%%%%%%%%%%%%%%%%%%%%%%%%%%%%%%%%%%%%%%%%%%
%                                             %
% KONIEC PERSONALIZACJI PRACY                 %
%                                             %
%%%%%%%%%%%%%%%%%%%%%%%%%%%%%%%%%%%%%%%%%%%%%%%

%%%%%%%%%%%%%%%%%%%%%%%%%%%%%%%%%%%%%%%%


%%%%%%%%%%%%%%%%%%%%%%%%%%%%%%%%%%%%%%%%%%%%%%%
%                                             %
% PROSZĘ NIE MODYFIKOWAĆ PONIŻSZYCH USTAWIEŃ! %
%                                             %
%%%%%%%%%%%%%%%%%%%%%%%%%%%%%%%%%%%%%%%%%%%%%%%



\documentclass[a4paper,twoside,12pt]{book}
\usepackage[utf8]{inputenc}                                      
\usepackage[T1]{fontenc}  
\usepackage{amsmath,amsfonts,amssymb,amsthm}
\usepackage[british,polish]{babel} 
\usepackage{indentfirst}
\usepackage{xurl}
\usepackage{xstring}
\usepackage{ifthen}



\usepackage{ifxetex}

\ifxetex
	\usepackage{fontspec}
	\defaultfontfeatures{Mapping=tex—text} % to support TeX conventions like ``——-''
	\usepackage{xunicode} % Unicode support for LaTeX character names (accents, European chars, etc)
	\usepackage{xltxtra} % Extra customizations for XeLaTeX
\else
	\usepackage{lmodern}
\fi



\usepackage[margin=2.5cm]{geometry}
\usepackage{graphicx} 
\usepackage{hyperref}
\usepackage{booktabs}
\usepackage{tikz}
\usepackage{pgfplots}
\usepackage{mathtools}
\usepackage{geometry}
\usepackage{subcaption}   % subfigures
\usepackage[page]{appendix} % toc,
\renewcommand{\appendixtocname}{Dodatki}
\renewcommand{\appendixpagename}{Dodatki}
\renewcommand{\appendixname}{Dodatek}

\usepackage{csquotes}
\usepackage[natbib=true,backend=bibtex,maxbibnames=99]{biblatex}  % kompilacja bibliografii BibTeXem
%\usepackage[natbib=true,backend=biber,maxbibnames=99]{biblatex}  % kompilacja bibliografii Biberem
\bibliography{biblio}

\usepackage{ifmtarg}   % empty commands  

\usepackage{setspace}
\onehalfspacing


\frenchspacing



%%%% TODO LIST GENERATOR %%%%%%%%%

\usepackage{color}
\definecolor{brickred}      {cmyk}{0   , 0.89, 0.94, 0.28}

\makeatletter \newcommand \kslistofremarks{\section*{Uwagi} \@starttoc{rks}}
  \newcommand\l@uwagas[2]
    {\par\noindent \textbf{#2:} %\parbox{10cm}
{#1}\par} \makeatother


\newcommand{\ksremark}[1]{%
{%\marginpar{\textdbend}
{\color{brickred}{[#1]}}}%
\addcontentsline{rks}{uwagas}{\protect{#1}}%
}

\newcommand{\comma}{\ksremark{przecinek}}
\newcommand{\nocomma}{\ksremark{bez przecinka}}
\newcommand{\styl}{\ksremark{styl}}
\newcommand{\ortografia}{\ksremark{ortografia}}
\newcommand{\fleksja}{\ksremark{fleksja}}
\newcommand{\pauza}{\ksremark{pauza `--', nie dywiz `-'}}
\newcommand{\kolokwializm}{\ksremark{kolokwializm}}
\newcommand{\cudzyslowy}{\ksremark{,,polskie cudzysłowy''}}

%%%%%%%%%%%%%% END OF TODO LIST GENERATOR %%%%%%%%%%%

\newcommand{\printCoauthor}{%		
    \StrLen{\FirstNameCoauthor}[\FNCoALen]
    \ifthenelse{\FNCoALen > 0}%
    {%
		{\large\bfseries\Coauthor\par}
	
		{\normalsize\bfseries \LeftId: \IdCoauthor\par}
    }%
    {}
} 

%%%%%%%%%%%%%%%%%%%%%
\newcommand{\autor}{%		
    \StrLen{\FirstNameCoauthor}[\FNCoALenXX]
    \ifthenelse{\FNCoALenXX > 0}%
    {\FirstNameAuthor\ \SurnameAuthor, \FirstNameCoauthor\ \SurnameCoauthor}%
	{\FirstNameAuthor\ \SurnameAuthor}%
}
%%%%%%%%%%%%%%%%%%%%%

\StrLen{\FirstNameCoauthor}[\FNCoALen]
\ifthenelse{\FNCoALen > 0}%
{%
\author{\FirstNameAuthor\ \SurnameAuthor, \FirstNameCoauthor\ \SurnameCoauthor}
}%
{%
\author{\FirstNameAuthor\ \SurnameAuthor}
}%

%%%%%%%%%%%% ZYWA PAGINA %%%%%%%%%%%%%%%
% brak kapitalizacji zywej paginy
\usepackage{fancyhdr}
\pagestyle{fancy}
\fancyhf{}
\fancyhead[LO]{\nouppercase{\it\rightmark}}
\fancyhead[RE]{\nouppercase{\it\leftmark}}
\fancyhead[LE,RO]{\it\thepage}


\fancypagestyle{tylkoNumeryStron}{%
   \fancyhf{} 
   \fancyhead[LE,RO]{\it\thepage}
}

\fancypagestyle{bezNumeracji}{%
   \fancyhf{} 
   \fancyhead[LE,RO]{}
}


\fancypagestyle{NumeryStronNazwyRozdzialow}{%
   \fancyhf{} 
   \fancyhead[LE]{\nouppercase{\autor}}
   \fancyhead[RO]{\nouppercase{\leftmark}} 
   \fancyfoot[CE, CO]{\thepage}
}


%%%%%%%%%%%%% OBCE WTRETY  
\newcommand{\obcy}[1]{\emph{#1}}
\newcommand{\english}[1]{{\selectlanguage{british}\obcy{#1}}}
%%%%%%%%%%%%%%%%%%%%%%%%%%%%%

% polskie oznaczenia funkcji matematycznych
\renewcommand{\tan}{\operatorname {tg}}
\renewcommand{\log}{\operatorname {lg}}

% jeszcze jakies drobiazgi

\newcounter{stronyPozaNumeracja}

%%%%%%%%%%%%%%%%%%%%%%%%%%% 
\newcommand{\printOpiekun}[1]{%		

    \StrLen{\Consultant}[\mystringlen]
    \ifthenelse{\mystringlen > 0}%
    {%
       {\large{\bfseries OPIEKUN, PROMOTOR POMOCNICZY}\par}
       
       {\large{\bfseries \Consultant}\par}
    }%
    {}
} 
%
%%%%%%%%%%%%%%%%%%%%%%%%%%%%%%%%%%%%%%%%%%%%%%
 
% Proszę nie modyfikować poniższych definicji!
\newcommand{\Author}{\FirstNameAuthor\ \MakeUppercase{\SurnameAuthor}} 
\newcommand{\Coauthor}{\FirstNameCoauthor\ \MakeUppercase{\SurnameCoauthor}}
\newcommand{\Type}{PROJEKT INŻYNIERSKI}
\newcommand{\Faculty}{Wydział Automatyki, Elektroniki i Informatyki} 
\newcommand{\Polsl}{Politechnika Śląska}
\newcommand{\Logo}{politechnika_sl_logo_bw_pion_pl.pdf}
\newcommand{\LeftId}{Nr albumu}
\newcommand{\LeftProgram}{Kierunek}
\newcommand{\LeftSpecialisation}{Specjalność}
\newcommand{\LeftSUPERVISOR}{PROWADZĄCY PRACĘ}
\newcommand{\LeftDEPARTMENT}{KATEDRA}
%%%%%%%%%%%%%%%%%%%%%%%%%%%%%%%%%%%%%%%%%%%%%%

%%%%%%%%%%%%%%%%%%%%%%%%%%%%%%%%%%%%%%%%%%%%%%%
%                                             %
% KONIEC USTAWIEŃ                             %
%                                             %
%%%%%%%%%%%%%%%%%%%%%%%%%%%%%%%%%%%%%%%%%%%%%%%




%%%%%%%%%%%%%%%%%%%%%%%%%%%%%%%%%%%%%%%%%%%%%%%
%                                             %
% MOJE PAKIETY, USTAWIENIA ITD                %
%                                             %
%%%%%%%%%%%%%%%%%%%%%%%%%%%%%%%%%%%%%%%%%%%%%%%

% Tutaj proszę umieszczać swoje pakiety, makra, ustawienia itd.

\usepackage{siunitx}
 
%%%%%%%%%%%%%%%%%%%%%%%%%%%%%%%%%%%%%%%%%%%%%%%%%%%%%%%%%%%%%%%%%%%%%
% listingi i fragmentu kodu źródłowego 
% pakiet: listings lub minted
% % % % % % % % % % % % % % % % % % % % % % % % % % % % % % % % % % % 

% biblioteka listings
\usepackage{listings}
\lstset{%
morekeywords={string,exception,std,vector},% słowa kluczowe rozpoznawane przez pakiet listings
language=Python,% C, Matlab, Python, SQL, TeX, XML, bash, ... – vide https://www.ctan.org/pkg/listings
commentstyle=\textit,%
identifierstyle=\textsf,%
keywordstyle=\sffamily\bfseries, %\texttt, %
%captionpos=b,%
tabsize=3,%
frame=lines,%
numbers=left,%
numberstyle=\tiny,%
numbersep=5pt,%
breaklines=true,%
escapeinside={@*}{*@},%
}

% % % % % % % % % % % % % % % % % % % % % % % % % % % % % % % % % % % 
% pakiet minted
%\usepackage{minted}

% pakiet wymaga specjalnego kompilowania:
% pdflatex -shell-escape main.tex
% xelatex  -shell-escape main.tex

%\usepackage[chapter]{minted} % [section]
%%\usemintedstyle{bw}   % czarno-białe kody 
%
%\setminted % https://ctan.org/pkg/minted
%{
%%fontsize=\normalsize,%\footnotesize,
%%captionpos=b,%
%tabsize=3,%
%frame=lines,%
%framesep=2mm,
%numbers=left,%
%numbersep=5pt,%
%breaklines=true,%
%escapeinside=@@,%
%}

%%%%%%%%%%%%%%%%%%%%%%%%%%%%%%%%%%%%%%%%%%%%%%%%%%%%%%%%%%%%%%%%%%%%%



%%%%%%%%%%%%%%%%%%%%%%%%%%%%%%%%%%%%%%%%%%%%%%%
%                                             %
% KONIEC MOICH USTAWIEŃ                       %
%                                             %
%%%%%%%%%%%%%%%%%%%%%%%%%%%%%%%%%%%%%%%%%%%%%%%



%%%%%%%%%%%%%%%%%%%%%%%%%%%%%%%%%%%%%%%%


\begin{document}
%\kslistofremarks

\frontmatter

%%%%%%%%%%%%%%%%%%%%%%%%%%%%%%%%%%%%%%%%%%%%%%%
%                                             %
% PROSZĘ NIE MODYFIKOWAĆ STRONY TYTUŁOWEJ!    %
%                                             %
%%%%%%%%%%%%%%%%%%%%%%%%%%%%%%%%%%%%%%%%%%%%%%%


%%%%%%%%%%%%%%%%%%  STRONA TYTUŁOWA %%%%%%%%%%%%%%%%%%%
\pagestyle{empty}
{
	\newgeometry{top=1.5cm,%
	             bottom=2.5cm,%
	             left=3cm,
	             right=2.5cm}
 
	\ifxetex 
	  \begingroup
	  \setsansfont{Calibri}
	   
	\fi 
	 \sffamily
	\begin{center}
	\includegraphics[width=50mm]{\Logo}
	 
	
	{\Large\bfseries\Type\par}
	
	\vfill  \vfill  
			 
	{\large\Title\par}
	
	\vfill  
		
	{\large\bfseries\Author\par}
	
	{\normalsize\bfseries \LeftId: \IdAuthor}

	\printCoauthor
	
	\vfill  		
 
	{\large{\bfseries \LeftProgram:} \Program\par} 
	
	{\large{\bfseries \LeftSpecialisation:} \Specialisation\par} 
	 		
	\vfill  \vfill 	\vfill 	\vfill 	\vfill 	\vfill 	\vfill  
	 
	{\large{\bfseries \LeftSUPERVISOR}\par}
	
	{\large{\bfseries \Supervisor}\par}
				
	{\large{\bfseries \LeftDEPARTMENT\ \Departament} \par}
		
	{\large{\bfseries \Faculty}\par}
		
	\vfill  \vfill  

    	
    \printOpiekun{\Consultant}
    
	\vfill  \vfill  
		
    {\large\bfseries  Gliwice \the\year}

   \end{center}	
       \ifxetex 
       	  \endgroup
       \fi
	\restoregeometry
}
  
%%%%%%%%%%%%%%%%%%%%%%%%%%%%%%%%%%%%%%%%%%%%%%%
%                                             %
% KONIEC STRONY TYTUŁOWEJ                     %
%                                             %
%%%%%%%%%%%%%%%%%%%%%%%%%%%%%%%%%%%%%%%%%%%%%%%  


\cleardoublepage

\rmfamily\normalfont
\pagestyle{empty}


%%% No to zaczynamy pisać pracę :-) %%%%

%--------------------------------------
%
%
% TODO: Streszczenie, słowa kluczowe, abstract, keywords
%
% Status: 0%
%--------------------------------------
\subsubsection*{Tytuł pracy} 
\Title

\subsubsection*{Streszczenie}  
(Streszczenie pracy – odpowiednie pole w systemie APD powinno zawierać kopię tego streszczenia.)

\subsubsection*{Słowa kluczowe} 
(2-5 slow (fraz) kluczowych, oddzielonych przecinkami)

\subsubsection*{Thesis title} 
\begin{otherlanguage}{british}
\TitleAlt
\end{otherlanguage}

\subsubsection*{Abstract} 
\begin{otherlanguage}{british}
(Thesis abstract – to be copied into an appropriate field during an electronic submission – in English.)
\end{otherlanguage}
\subsubsection*{Key words}  
\begin{otherlanguage}{british}
(2-5 keywords, separated by commas)
\end{otherlanguage}




%%%%%%%%%%%%%%%%%% SPIS TRESCI %%%%%%%%%%%%%%%%%%%%%%
% Add \thispagestyle{empty} to the toc file (main.toc), because \pagestyle{empty} doesn't work if the TOC has multiple pages
\addtocontents{toc}{\protect\thispagestyle{empty}}
\tableofcontents

%%%%%%%%%%%%%%%%%%%%%%%%%%%%%%%%%%%%%%%%%%%%%%%%%%%%%
\setcounter{stronyPozaNumeracja}{\value{page}}
\mainmatter
\pagestyle{empty}

\cleardoublepage

\pagestyle{NumeryStronNazwyRozdzialow}

%%%%%%%%%%%%%% wlasciwa tresc pracy %%%%%%%%%%%%%%%%%

%--------------------------------------
%
%
% TODO: cel pracy (99.9%), zakres pracy(99%), zwięzła charakterystyka rozdziałów
%
% Status: 80%
%--------------------------------------
\chapter{Wstęp}
\label{ch:wstep}

\section{Wprowadzenie i osadzenie w dziedzinie}
% wprowadzenie w problem/zagadnienie
Śledzenie ruchu gałek ocznych (ang. eye-tracking), zwane również jako okulografia, jest techniką, która badana jest od ponad stu lat. Istotność tej techniki wynika z faktu, że ruchy gałek ocznych są ściśle związane z procesami poznawczymi, takimi jak uwaga, percepcja, pamięć, czy procesy decyzyjne. Skupiając wzrok na danym punkcie, umieszczamy go w centralnym obszarze naszego pola widzenia które charakteryzuje się największą rozdzielczością, co pozwala na dokładne analizowanie szczegółów. Ten fakt wpływa także na proces skupienia -- gdy koncentrujemy się na danym obiekcie lub obszarze, skupiamy na nim wzrok (często wystarczy jedynie krótki moment).

Możliwość rejestrowania ruchów oczu pozwala na zrozumienie w jaki sposób obserwator eksploruje otaczający go świat. Posiadając tą wiedzę możliwe jest wyciągnięcie wniosków na temat tego co jest interesujące lub istotne dla obserwatora, jakie emocje się z tym wiążą, czy nawet jakie procesy poznawcze zachodzą w jego umyśle, czy rozumie on to co widzi. Nie trudnym jest zauważyć jak cenne mogą być te informacje w szerokim spektrum dziedzin.

% osadzenie problemu w dziedzinie
Okulografia odgrywa kluczową rolę w psychologi poznawczej, psychologi społecznej, neurobiologii, marketingu, czy medycynie. W psychologii poznawczej ruch oczu jest ściśle związany z pamięcią, podejmowaniem decyzji, obciążeniem poznawczym i uczeniem się asocjacyjnym. W psychologii społecznej eye-tracking pozwala na wgląd w zachowania społeczne i ich analizę, co pozwala badać empatię, prospołeczność, czy fobie społeczne \cite{bib:tobii-main}. W neurobiologii bada się powiązania ruchu oczu ze szlakami neuronowymi odpowiedzialnymi za podejmowane akcje i procesy myślowe, dając możliwość w diagnozach i wsparciu osób dotkniętych chorobą Parkinsona \cite{bib:tobii-parkinson}, Alzheimera \cite{bib:tobii-alzheimer}, a także autyzmem czy łagodnym upośledzeniem funkcji poznawczych \cite{bib:tobii-autyzm}. W medycynie okulografia pozwala na diagnozę między innymi oczopląsu, który może być objawem np. stwardnienia rozsianego, uszkodzenia śródmózgowia lub urazu ucha wewnętrznego \cite{bib:Diagnostyka-oczoplas}. Niezaprzeczalnie technika ta jest niezwykle cenna i efektywna na wielu płaszczyznach naukowych i praktycznych.

\section{Cel oraz zakres pracy}
% cel pracy
Celem niniejszej pracy jest stworzenie narzędzia które pozwoli na uniwersalne śledzenie ruchu gałek ocznych, bez zdefiniowanej docelowej grupy użytkowników. System ten powinien być prosty w obsłudze i niewymagający zaawansowanej technologii, ale nie ograniczający w razie potrzeby bardziej zaawansowanych użytkowników, prezentując użyteczność zarówno dla naukowców, studentów, jak i hobbystów. Tego typu narzędzie pozwoli na eksplorację danych eye-trackingowych, otwierając możliwości na dalszą analizę i interpretację wyników, a także zapewni solidną podstawę w razie potrzeby modyfikacji lub rozbudowy systemu w bardziej ukierunkowany sposób.

% zakres pracy
Praca ta skupia się przede wszystkim na samym procesie wykrywania źrenic i śledzenia ruchu gałek ocznych na ich podstawie oraz implementacji systemu wizyjnego, który pozwoli na zapis, wizualizację i potencjalną dalszą analizę danych zebranych przez kamerę, w tym także kamerę internetową. Całość zrealizowana jest w języku Python, z wykorzystaniem bibliotek takich jak OpenCV, NumPy, Pandas oraz Matplotlib, a wszelkie prezentowane dane zostały uchwycone przy użyciu wbudowanej kamery laptopa. Praca nie będzie obejmować rozległej analizy zebranych danych, ani eksperymentów przeprowadzonych z użyciem owego systemu. Nie została także stworzona grupa testowa, więc większość testów przeprowadzone zostały na autorze pracy.

\begin{itemize}
\item wprowadzenie w problem/zagadnienie
\item osadzenie problemu w dziedzinie
\item cel pracy
\item zakres pracy
\item zwięzła charakterystyka rozdziałów
\item jednoznaczne określenie wkładu autora, w przypadku prac wieloosobowych – tabela z autorstwem poszczególnych elementów pracy
\end{itemize}



%--------------------------------------
%
%
% TODO: sformułowanie problemu (99% ), state of the art, studia literaturowe
%
% Status: 43%
%--------------------------------------
\chapter{[Analiza tematu]}

% =========== moja część ===========
% sformułowanie problemu
\section{Sformułowanie problemu}

Widzeniem plamkowym (fovealnym) nazywamy wspomniane już wcześniej zjawisko, że widzimy w największej rozdzielczości jedynie centrum pola widzenia. Wynika to z budowy siatkówki, która posiada plamkę żółtą (fovea) -- jest to niewielki obszar, który charakteryzuje się największym zagęszczeniem czopków. Taka budowa siatkówki sprawia, że nieustannie poruszamy oczami. Skupiając wzrok na danym punkcie, musimy być gotowi na to, że cała reszta pola widzenia straci dla nas na ostrości. Jest to proces, który przypomina filtracje wielu otaczających nas informacji, do paru, które mają w tym momencie znaczenie.

\subsection{Uwaga wzrokowa}

Uwaga wzrokowa może zostać opisana przez idiomy ,,gdzie'' i ,,co'', które pozwalają na zobrazowanie jej selektywnej natury. ,,Gdzie'' to proces wizualnego wyszukiwania i wyboru lokalizacji, która zwróciła naszą uwagę, w celu dokładniejszego zbadania. Istotnym aspektem tego procesu selekcji jest widzenie peryferyjne, czyli takie obejmujące obszar oddalony od naszego punktu skupienia, ale wciąż pozwalający na wyodrębnienie kształtów i ruchu, co w pewien sposób prowadzi nasze centralne spojrzenie. Przykładem może być spojrzenie przez okno -- w pierwszym momencie nasz wzrok kierowany jest na wyraźne kształty, światła czy nagłe ruchy jak np. lądujący ptak. 

,,Co'' można nazwać odwrotnością ,,gdzie'' -- jest to proces szczegółowego badania danego obszaru, charakteryzują go takie zjawiska jak fiksacja, czyli stabilizacja wzroku na danym punkcie oraz ruchy sakadowe, czyli szybkie mimowolne ruchy oczu pomiędzy kolejnymi punktami, które pozwalają na obserwacje. Całość tworzy kanał percepcyjny o ograniczonym zasięgu przestrzennym, obejmujący interesujący obszar, budując w ten sposób naszą świadomość i zrozumienie celu naszego spojrzenia. Wracając do poprzedniego przykładu, po zauważeniu ptaka, nasze spojrzenie skupi się na nim (fiksacja), a ruchy sakadowe oczu pozwolą na dokładne zbadanie jego kształtu, koloru i detali. To właśnie połączenie ,,gdzie'' i ,,co'' pozwala na głębsze zrozumienie otaczającego nas świata, a owe ścieżki skanowania dają nam 
wgląd w proces poznawczy badanej osoby, jako że pomiar widzenia fovealnego w czasie odzwierciedla chwilowe i jawne skupienie uwagi wzrokowej obserwatora. 

\subsection{Poszukiwanie wzrokowe}

Tutaj warto wspomnieć o poszukiwaniu wzrokowym, jest to proces aktywnego przeszukiwania pola widzenia w celu znalezienia konkretnego celu, a częścią tego procesu jest etap przeduwagowy. Etap ten równolegle analizuje duży obszar widzenia jednocześnie, dzieje się to automatycznie, nieświadomie i nie wymaga widzenia fovealnego. Co charakteryzuje ten proces to umiejętność rozpoznania czterech podstawowych cech: kolor, rozmiar, orientacja oraz obecność i/lub kierunek ruchu. Etap przeduwagowy jest pierwszym krokiem poszukiwania wzrokowego i obejmuje dużą część poszukiwania typu równoległego (np. próba zauważenia nagłego ruchu spadającej gwiazdy) i relatywnie niewielką część poszukiwania typu seryjnego (np. przeszukiwanie obiektów na stole w poszukiwaniu kluczy), które wymaga uwagi i przenoszenia wzroku od obiektu do obiektu. 

\subsection{Analiza ruchów oczu}

Powyższe rozważania pozwalają przejść do głównego nurtu tematu śledzenia ruchu gałek ocznych, czyli analiza ruchów oczu. Rozróżniamy pięć podstawowych typów ruchów oczu: wspomniane już wcześniej sakadyczne, wergencyjne, przedsionkowe, płynne podążanie (smooth pursuit) oraz oczopląs fizjologiczny, który jest naturalny dla zdrowej osoby i często występuje podczas fiksacji. Ruchy te można podzielić na dobrowolne, mimowolne i odruchowe, a sygnały je kontrolujące pochodzą z obszarów korowych mózgu. Do opisania tych procesów można posłużyć się modelowaniem matematycznym. Do zrozumienia jawnej uwagi wzrokowej wystarczy modelowanie trzech typów ruchów: fiksacji, która pokazuje chęć utrzymania wzroku na stacjonarnym obiekcie, sakad, które mogą wskazywać na chęć zmiany punktu uwagi oraz płynnych pościgów, które podobnie do fiksacji, pozwalają na śledzenie obiektu, ale ruchomego. 

\subsubsection{Ruchy sakadowe}

Sygnał sakad można opisać jako funkcja impuls/skok, gdzie impuls na wejściu reprezentuje prędkość, a skok pozycję, impuls jest przepuszczany przez filtr, który przekształca go w skok. Prostą reprezentacją ruchu sakadowego jest filtr liniowy różniczkujący, który dokonuje potrzebnej konwersji informacji prędkość na przemieszczenie. Wzór owego filtra w dziedzinie czasu można zapisać następująco

\begin{align*}
	x_t &= g_0 \cdot s_t + g_1 \cdot s_{t-1} + g_2 \cdot s_{t-2} + \ldots
\end{align*}

Wyjście $x_t$ zależy od bieżącego wejścia $s_t$ oraz jego poprzednich wartości $(s_{t-1}, s_{t-2}, \ldots)$ ważonych odpowiednimi współczynnikami filtra $(g_0, g_1, g_2, \ldots)$. W zapisie sumarycznym wzór prezentuje się następująco

\begin{align*}
	x_t &= \sum_{k=0}^{\infty} g_k \cdot s_{t-k}
\end{align*}

Filtr Haara jest jednym z przykładów filtrów który przybliża różniczkowanie, jest to filtr o długości 2, czyli operujący jedynie na dwóch kolejnych próbkach sygnału wejściowego $(s_t, s_{t-1})$, a przybliża on pierwszą pochodną, przyjmując, że prędkość zmian sygnału jest stała w czasie trwania dwóch próbek. Współczynniki filtra są równe $g_0 = 1$ oraz $g_1 = -1$, co oznacza że różniczkowanie jest zrealizowane poprzez różnicę między dwiema kolejnymi próbkami sygnału. W związku z tym transmitancję filtru Haara można zapisać jako

\begin{align*} %!!!!!!!!!!!!!!!!!!!!!!!!!!!!TUTAJ JEST COŚ ŹLE Z TYM RÓWNANIEM%!!!!!!!!!!!!!!!!!!!!!!!!!!!!
	x_t &= g_0 \cdot s_t + g_1 \cdot s_{t-1} \\
	x_t &= 1 \cdot s_t - 1 \cdot s_{t-1} \\
	x_t &= s_t - s_{t-1} \\
	\mathcal{Z}\{x_t\} &= \mathcal{Z} \{s_t - s_{t-1}\} \\
	X(z) &= (1 - z) \cdot S(z) \\ % POWINNO BYĆ CHYBA (1 - z^{-1}) \cdot S(z)
	\frac{X(z)}{S(z)} &= 1 - z
\end{align*}   %!!!!!!!!!!!!!!!!!!!!!!!!!!!!TUTAJ JEST COŚ ŹLE Z TYM RÓWNANIEM%!!!!!!!!!!!!!!!!!!!!!!!!!!!!

Diagram omawianego modelu przedstawiony jest na rysunku \ref{fig:model-sakad}.

\begin{figure}[h]
	\centering
	\includegraphics[width=0.5\textwidth]{pic/modele/model_sakad.png}
	\caption{Diagram modelu ruchów sakadowych z filtrem liniowym różniczkującym.}
	\label{fig:model-sakad}
\end{figure}

\subsubsection{Płynne podążanie}

Płynne podążanie występuje gdy obserwator śledzi obiekt w ruchu, ruch ten oczywiście nie może być zbyt gwałtowny. Tego typu śledzenie ruchu jest przykładem systemu sterowania z ujemnym sprzężeniem zwrotnym. Do modelowania owych ruchów używana jest prosta pętla, którą można opisać nastepujacym równaniem w dziedzinie czasu

\begin{align*}
	h \cdot (s_t - x_t) &= x_{t+1}
\end{align*}

Wejście $s_t$ to pozycja celu, wyjście $x_t$ to pożądana pozycja oka. Można zauważyć, że w tym przypadku nie jest wymagana transformacja informacji wejściowej na wyjściową, a jedynie zmodyfikowanie jej wartości, w związku z tym $h$ jest liniowym niezmiennym w czasie filtrem, czyli po prostu wzmocnieniem systemu. Przeprowadźmy teraz transformację Z na powyższym równaniu by uzyskać transmitancję systemu

\begin{align*} %!!!!!!!!!!!!!!!!!!!!!!!!!!!!TUTAJ JEST COŚ ŹLE Z TYM RÓWNANIEM%!!!!!!!!!!!!!!!!!!!!!!!!!!!!
	\mathcal{Z}\{h \cdot (s_t - x_t)\} &= \mathcal{Z}\{x_{t+1}\} \\
	H(z) \cdot (S(z) - X(z)) &=  X(z) \\ % POWINNO BYĆ CHYBA z \cdot X(z)
	H(z) \cdot S(z) - H(z) \cdot X(z) &=  X(z) \\
	H(z) \cdot S(z) &= (H(z) + 1) \cdot X(z) \\
	\frac{X(z)}{S(z)} &= \frac{H(z)}{H(z) + 1}
\end{align*}   %!!!!!!!!!!!!!!!!!!!!!!!!!!!!TUTAJ JEST COŚ ŹLE Z TYM RÓWNANIEM%!!!!!!!!!!!!!!!!!!!!!!!!!!!!

W ten sposób sygnał z receptorów oka służy za błąd, który następnie jest kompensowany w celu utrzymania obrazu w przestrzeni widzenia plamkowego. Diagram modelu przedstawiony jest na rysunku \ref{fig:model-smooth}.

\begin{figure}[h]
	\centering
	\includegraphics[width=0.6\textwidth]{pic/modele/model_smooth.png}
	\caption{Diagram modelu płynnego podążania z liniowym sprzężeniem zwrotnym.}
	\label{fig:model-smooth}
\end{figure}

\subsubsection{Fiksacja}

Można zauważyć, że fiksacja jest procesem podobnym do płynnego podążania, z tą różnicą, że w tym przypadku obiekt jest nieruchomy. Jednakże proces ten nie da się bezpośrednio z sobą porównać i najprawdopodobniej nie ma wspólnego obwodu neuronalnego. Nasze komórki systemu wzrokowego są fizjologicznie wrażliwe na ruch, gdyby dany obiekt został unieruchomiony względem siatkówki, po krótkim czasie widzenie zaniknie. Powoduje to konieczność mikrosakad i innych mimowolnych, drobnych ruchów oczu. Można więc uznać, że model fiksacji jest podobny do modelu płynnego podążania, który próbuje utrzymać pozycję oka na danym punkcie, a mikrosakady i inne ruchy można uznać za szum w systemie kontrolnym, który można wyrazić jako $e_t = x_t - s_t$. Jest to losowa fluktuacja wokół punktu fiksacji, a jej wartość średnia pozostaje stała.

Ruch gałek ocznych i związane z nim widzenie jest bardzo szerokim tematem interdyscyplinarnym, na potrzeby tego projektu powyższe sformułowanie problemu, chociaż uproszczone i nie wyczerpujące, pozwala na zrozumienie w pełni istoty prezentowanego systemu.

% Osadzenie tematu w kontekście aktualnego stanu wiedzy

\section{Osadzenie tematu w kontekście aktualnego stanu wiedzy}

\subsection{Metody śledzenia ruchu oczu: przegląd historyczny i alternatywy}

Pomiar ruchu oczu był realizowany już w latach 70. XX wieku, nic dziwnego więc, że od tego momentu powstały różne techniki pozwalające na zebranie owych danych. Należy podkreślić, że metody te można podzielić na mierzące ruch oczu względem głowy, oraz takie które mierzą orientacje oczu w przestrzeni, czyli ,,punkt spojrzenia''. 

Najstarszą z nich jest elektro-okulografia (EOG ang. \english{electrooculography}), wciąż wykorzystywana np. w badaniach klinicznych. Metoda ta polega na umieszczeniu elektrod na skórze twarzy wokół oczu i pomiarze różnicy potencjałów. Metoda ta z założenia mierzy ruch oczu względem głowy, ale można rozszerzyć ją o pomiar ruchu głowy dając możliwość wyliczenia punktu spojrzenia. 

Najdokładniejszą metodą pomiaru ruchu oczu polega na umieszczeniu soczewki kontaktowej bezpośrednio na gałce ocznej. Soczewka ta powinna być odpowiednio duża, by objąć powierzchnie zarówno rogówki jak i twardówki, a na jej powierzchni umieszczany jest albo obiekt optyczny, których zdaniem jest dokładne odbijanie światła lub dostarczenie wyraźnych kształtów potrzebnych do śledzenia, albo cewkę wykonaną z drutu, która poruszając się w polu elektromagnetycznym pozwala na wykonanie pomiaru. Metoda ta jest bardzo dokładna, ale niezwykle inwazyjna i nieprzyjemna dla badanego.

Wideo-okulografia i foto-okulografia tworzą razem szeroką grupę metod, które opierają się na analizie wyróżniających się cech oka podczas jego obrotu, takich jak: kształt źrenica, pozycja granicy tęczówki i twardówki oraz odbicie światła od rogówki (często podczerwonego), którego źródło zostało umieszczone bardzo blisko oka. Metody te są nieinwazyjne, ale same w sobie nie pozwalają na określenie punktu spojrzenia, więc w celu wyznaczenia owego punktu często stosuje się unieruchomienie głowy osoby badanej, wyznaczenie punktu odniesienia np. przez odbicia światła od powierzchni oka, lub odpowiednią kalibrację np. prosząc osobę badaną o utrzymanie wzroku na danym punkcie ekranu.

\subsection{Analiza wideo z wykorzystaniem źrenicy i odbicia rogówkowego}

Opisane metody wymagają unieruchomienia głowy (przez różnego rodzaju podpórki pod brodę lub głowę, albo nawet belki nagryzowe) lub zastosowanie dodatkowego sprzętu mierzącego ruch głowy, by wyznaczyć punkt spojrzenia. Jest to główna wada tych metod, jako że miejsce skupienia uwagi daje istotne informacje, które są najczęściej pożądane przez użytkowników okulografów . Dlatego metoda opierająca się na analizie obrazu wideo z kombinacją wykrycia źrenicy i odbicia rogówkowego, która pozwala na określenie punktu spojrzenia z dużą dokładnością, jest jedną z najpopularniejszych wyborów, zarówno w świecie naukowym jak i komercyjnym. System prezentowany w tej pracy można zaliczyć do wideo-okulografii, z możliwością włączenia trybu z pomiarem odbicia rogówkowego, więc metoda ta zostanie omówiona szerzej od pozostałych.

Wykrywanie punktu spojrzenia metodą detekcji źrenicy i odbicia rogówkowego wymaga kilku kluczowych elementów. Potrzebna jest kamera wideo, która zarejestruje aktualny stan gałek ocznych oraz sprzęt, który pozwoli na przetworzenie obrazu i ostatecznie wykrycie interesujących cech, najlepiej w czasie rzeczywistym. Aktualnie oba wymienione elementy stają się coraz tańsze i bardziej wydajne, tak naprawdę większość osób posiada je w swoim smartfonie, co sprawia, że technika ta staje się bardziej dostępna dla szerokiego grona użytkowników. Aparatura ta jest także coraz bardziej miniaturyzowana, dzięki czemu aktualnie dostępne są systemy montowane na głowie, jak i montowane na biurku, które działają na tej samej zasadzie, różniąc się zasadniczo jedynie wielkością. Warto jednak zaznaczyć, że systemy montowane na głowie zazwyczaj posiadają także dodatkową kamerę wyznaczającą tak zwany POV (ang. point of view, czyli punkt widzenia) i spojrzenie jest monitorowane względem tego obrazu, gdzie te montowane na biurku zazwyczaj mierzą spojrzenie względem pewnej powierzchni np. monitora. Należy omówić także źródło światła, którego odbicie rejestrowane jest podczas detekcji. Możliwe jest wykorzystanie punktowego źródła światła białego o odpowiednim natężeniu, jednakże może to prowadzić do dyskomfortu przy użytkowaniu lub nawet chwilowego oślepienia osoby badanej, wytrącając ją z stanu skupienia. W związku z tym najczęściej wykorzystywanym są źródła światła podczerwonego, które znajdują się w większości poza widzialnym spektrum nie irytując oka, nie zakłócając procesu badania i pozwalając na użycie wyższego natężenia, ale wymagają specjalnych kamer, które są w stanie rejestrować światło IR (ang. infrared). 

Omawiane odbicie rogówkowe nazywane jest obrazem Purkinjego, budowa oka sprawia, że pojawiają się cztery odbicia, pierwsze to odbicie od zewnętrznej powierzchni rogówki (najbardziej widoczne), drugie od wewnętrznej powierzchni rogówki, trzecie od zewnętrznej powierzchni soczewki, a czwarte od wewnętrznej powierzchni soczewki. By poprawnie wyznaczyć punkt spojrzenia, wymagane jest wyznaczenie dwóch punktów odniesienia, pierwszy określający obrót oczu w oczodole, a drugi określający stałe położenie względem oczu. Pierwszym punktem zazwyczaj jest środek źrenicy, a drugim najczęściej jest pierwszy obraz Purkinjego, powstały przez źródło światła ustabilizowane względem głowy, lub powierzchni badanej. Przy nieruchomej głowie, ruch oczu odczytywany będzie jako zmieniająca się różnica pomiędzy tymi dwoma punktami, w odwrotnym przypadku, gdy głowa poruszy się, a oczy pozostaną w fiksacji na danym punkcie, dla systemu montowanego na biurku różnica ta pozostanie stała, a dla systemu montowanego na głowie, różnica będzie zmieniać się proporcjonalnie do zmiany punktu widzenia. Istnieją także okulografy piątej generacji z technologią DPI (ang. Dual Purkinje Image), mierzą one dodatkowo czwarty obraz Purkinjego, dzięki temu, mogą rozróżniać ruchy translacyjne oka od ruchów rotacyjnych, jako że przy translacji obrazy Purkinjego poruszają się o tą samą odległość, a przy rotacji zmieniają swoje rozdzielenie, co zwiększa precyzję pomiaru, ale wymaga bardziej skomplikowanego oprogramowania i unieruchomienie głowy może okazać się konieczne.

\section{Studia literaturowe}

Powyższe sformułowanie problemu i jego osadzenie w kontekście aktualnego stanu wiedzy zostało przedstawione w oparciu o książkę ,,Eye Tracking Methodology: Theory and Practice'' autorstwa Andrew T. Duchowskiego \cite{bib:eye-tracking-methodology}. Książka ta dogłębnie omawia temat śledzenia ruchu oczu, pozwalając na zgłębienie zarówno podstaw procesów wzrokowych w kontekstach różnych dziedzin naukowych, a także szeroko omawia praktykę pomiaru ruchu oczu. Powyższa analiza tematu skupia się jedynie na najistotniejszych aspektach, które pozwolą na zrozumienie istoty prezentowanego systemu, a wiedzę na ich temat można poszerzyć lekturą wspomnianej książki.

Eye-tracking staje się coraz bardziej dostępny, co sprawia że technologia ta znajduje zastosowanie w wielu obszarach, zarówno w badaniach naukowych jak i do użytku konsumenckiego czy komercyjnego. Część producentów owych systemów udostępnia publikacje naukowe, w których zostały one wykorzystane, a ich analiza przedstawia różne rozwiązania omawianego tematu.

Jednym z przedstawicieli firm skupiających się na analizie zachowań ludzkich jest iMotions, wyróżniając się modularnym systemem z centralnym hubem. Firma ta oferuje biosensory w formie modułów, pozwalające na np. śledzenie oczu, analizę wyrazu twarzy, rejestrowanie aktywności elektrycznej mózgu i rejestrowanie aktywności serca. Dane zebrane przez różne czujniki są integrowane w jednym oprogramowaniu co pozwala badaczom na prostą synchronizację i analizę danych w jednym systemie. Jednak mimo tak obszernego asortymentu biosensorów, to właśnie moduł śledzenia wzroku na ekranie (montowany na biurku) był wykorzystywany najczęściej w badaniach naukowych, z czego eye-tracking był także najczęściej wykorzystywany w systemach jednomodułowych \cite{bib:iMotions-2023-report}, wskazuje to jak uniwersalnym i podstawowym narzędziem jest eye-tracker w badaniach behawioralnych i nie tylko. 

Jednym z przeprowadzonych badań z użyciem technologii iMotions było ,,Using Psychophysiological Data to Facilitate Reflective Conversations with Children about their Player Experiences''. Podobne badania były przeprowadzane wielokrotnie w stosunku do osób dorosłych, jednak w przypadku grupy badanej składającej się z dzieci, badacze mogą napotkać znaczny problem z odpowiednim zrozumieniem wywiadu, w którym uczestnicy dzielą się swoimi wrażeniami z przeprowadzonej rozgrywki. Omawiane badanie miało głównie na celu sprawdzenie czy zebrane dane z biosensorów są wstanie wspomóc refleksje dzieci na temat ich doświadczenia w grze i czy dzieci napotkają problemy z interpretacją prezentowanych danych. Odnoszenie się do stanu psychofizjologicznego może zminimalizować problem z komunikacją pomiędzy dzieckiem, które posiada ograniczone umiejętności werbalne, a badaczem. Do śledzenia ruchu gałek ocznych na ekranie zastosowano okulograf Smart Eye AI-X o częstotliwości próbkowania $60 \unit{\hertz}$, posługujący się techniką detekcji źrenic i odbicia rogówkowego, pozwalając na sporą swobodę ruchu głowy w przestrzeni $35 \unit{\centi\metre}$ na $30 \unit{\centi\metre}$, jednocześnie posiadając dokładność (różnica pomiędzy rzeczywistą pozycją spojrzenia a pozycją spojrzenia zarejestrowaną przez okulograf) \ang{0.5} i precyzję (średnia kwadratowa punktów mierzonych w jednej pozycji oka) \ang{0.1}, zwracając dane wyjściowe binokularowe z wskaźnikiem jakości zawierające: punkt spojrzenia, średnicę źrenicy oraz znacznik czasowy \cite{bib:iMotions-smart-eye-ai-x}. Oprócz tego zastosowano moduł rejestrujący aktywność sercową, analizujący wyrazy twarzy oraz mierzący reakcję skórno-galwaniczną. Badanie składało się z dwóch etapów, wpierw zebrano dane w laboratorium podczas gdy badane dzieci grały w dwie gry oraz podczas wywiadów bezpośrednio po zakończonej rozgrywce, a następnie prezentowano badanym momenty z pierwszego etapu z uwzględnieniem danych z biosensorów i zadawano pytania o ich doświadczenia. Badania wskazały na to, że dzieci są wstanie zrozumieć i odpowiednio odnieść się do prezentowanych danych psychofizjologicznych, a oparcie się o nie może pomóc w zwerbalizowaniu swojego doświadczenia i informacji zwrotnej przez badane dzieci. Dzieci nie miały problemu z zrozumieniem większości pomiarów za wyjątkiem reakcji skórno-galwanicznej, jednakże dane eye-trackingowe były swego rodzaju wyjątkiem, jako że nie prezentowały bezpośrednio doświadczenia i uczuć związanych z danym wydarzeniem w grze, a raczej dawały wgląd w strategię objęta przez dziecko zmagające się z aktualnym wyzwaniem, mimo tego prezentowane śledzenie punktu spojrzenia było naturalne w interpretacji przez badanych \cite{bib:iMotions-children}. Publikacja ta pokazuje jak okulografia może zostać skutecznie użyta w badaniach naukowych, ale przede wszystkim przedstawia istotność tej techniki w zastosowaniach komercyjnych np. testując doświadczenia graczy przy produkcji gier wideo dla różnych grup wiekowych.






\begin{itemize}
\item sformułowanie problemu
\item osadzenie tematu w kontekście aktualnego stanu wiedzy (\english{state of the art}) o poruszanym problemie
\item  studia literaturowe \cite{bib:artykul,bib:ksiazka,bib:konferencja,bib:internet} -  opis znanych rozwiązań (także opisanych naukowo, jeżeli problem jest poruszany w publikacjach naukowych), algorytmów, 
\end{itemize}


Wzory  
\begin{align}
y = \frac{\partial x}{\partial t}
\end{align}
jak i pojedyncze symbole $x$ i $y$  składa się w trybie matematycznym.


%%%%%%%%%%%%%%%%%%%%%%%%





%--------------------------------------
%
%
% TODO: wymagania funkcjonalne i niefunkcjonalne, przypadki użycia, opis narzędzi itp., metodyka pracy
%
% Status: 0%
%--------------------------------------
\chapter{Wymagania i narzędzia}
\label{ch:wymagania-i-narzedzia}

\begin{itemize}
\item wymagania funkcjonalne i niefunkcjonalne
\item przypadki użycia (diagramy UML) -- dla prac, w których mają zastosowanie
\item opis narzędzi, metod eksperymentalnych, metod modelowania itp.
\item metodyka pracy nad projektowaniem i implementacją -- dla prac, w których ma to zastosowanie
\end{itemize}


%--------------------------------------
%
%
% TODO: wymagania sprzętowe i programowe, sposób instalacji, aktywacji, kategorie użytkowników, sposób obsługi, administracja systemem, kwestie bezpieczeństwa, przykład działania, scenariusze korzystania z systemu
%
% Status: 0%
%--------------------------------------
\chapter{[Właściwy dla kierunku -- np. Specyfikacja zewnętrzna]}
\label{ch:04}

Jeśli „Specyfikacja zewnętrzna”:
\begin{itemize}
\item  wymagania sprzętowe i programowe
\item  sposób instalacji
\item  sposób aktywacji
\item  kategorie użytkowników
\item  sposób obsługi
\item  administracja systemem
\item  kwestie bezpieczeństwa
\item  przykład działania
\item  scenariusze korzystania z systemu (ilustrowane zrzutami z ekranu lub generowanymi dokumentami)
\end{itemize}

%%%%%%%%%%%%%%%%%%%%%
%% RYSUNEK Z PLIKU
%
%\begin{figure}
%\centering
%\includegraphics[width=0.5\textwidth]{./politechnika_sl_logo_bw_pion_pl.pdf}
%\caption{Podpis rysunku zawsze pod rysunkiem.}
%\label{fig:etykieta-rysunku}
%\end{figure}
%Rys. \ref{fig:etykieta-rysunku} przestawia …
%%%%%%%%%%%%%%%%%%%%%
%
%%%%%%%%%%%%%%%%%%%%%
%% WIELE RYSUNKÓW 
%
%\begin{figure}
%\centering
%\begin{subfigure}{0.4\textwidth}
%    \includegraphics[width=\textwidth]{./politechnika_sl_logo_bw_pion_pl.pdf}
%    \caption{Lewy górny rysunek.}
%    \label{fig:lewy-gorny}
%\end{subfigure}
%\hfill
%\begin{subfigure}{0.4\textwidth}
%    \includegraphics[width=\textwidth]{./politechnika_sl_logo_bw_pion_pl.pdf}
%    \caption{Prawy górny rysunek.}
%    \label{fig:prawy-gorny}
%\end{subfigure}
%
%\begin{subfigure}{0.4\textwidth}
%    \includegraphics[width=\textwidth]{./politechnika_sl_logo_bw_pion_pl.pdf}
%    \caption{Lewy dolny rysunek.}
%    \label{fig:lewy-dolny}
%\end{subfigure}
%\hfill
%\begin{subfigure}{0.4\textwidth}
%    \includegraphics[width=\textwidth]{./politechnika_sl_logo_bw_pion_pl.pdf}
%    \caption{Prawy dolny rysunek.}
%    \label{fig:prawy-dolny}
%\end{subfigure}
%        
%\caption{Wspólny podpis kilku rysunków.}
%\label{fig:wiele-rysunkow}
%\end{figure}
%Rys. \ref{fig:wiele-rysunkow} przestawia wiele ważnych informacji, np. rys. \ref{fig:prawy-gorny} jest na prawo u góry.
%%%%%%%%%%%%%%%%%%%%%


 
\begin{figure}
\centering
\begin{tikzpicture}
\begin{axis}[
    y tick label style={
        /pgf/number format/.cd,
            fixed,   % po zakomentowaniu os rzednych jest indeksowana wykladniczo
            fixed zerofill, % 1.0 zamiast 1
            precision=1,
        /tikz/.cd
    },
    x tick label style={
        /pgf/number format/.cd,
            fixed,
            fixed zerofill,
            precision=2,
        /tikz/.cd
    }
]
\addplot [domain=0.0:0.1] {rnd};
\end{axis} 
\end{tikzpicture}
\caption{Podpis rysunku po rysunkiem.}
\label{fig:2}
\end{figure}



%--------------------------------------
%
%
% TODO: przedstawienie idei, architektura systemu, opis struktur danych, komponenty itp. (jeśli występują), przegląd ważniejszych algorytmów (jeśli występują), szczegóły implementacji wybranych fragmentów, diagramy UML
%
% Status: 0%
%--------------------------------------
\chapter{[Właściwy dla kierunku -- np. Specyfikacja wewnętrzna]}
\label{ch:05}


Jeśli „Specyfikacja wewnętrzna”:
\begin{itemize}
\item przedstawienie idei
\item architektura systemu
\item opis struktur danych (i organizacji baz danych)
\item komponenty, moduły, biblioteki, przegląd ważniejszych klas (jeśli występują)
\item przegląd ważniejszych algorytmów (jeśli występują)
\item szczegóły implementacji wybranych fragmentów, zastosowane wzorce projektowe
\item diagramy UML
\end{itemize}

% % % % % % % % % % % % % % % % % % % % % % % % % % % % % % % % % % % 
% Pakiet minted wymaga importu: \usepackage{minted}                 %
% i specjalnego kompilowania:                                       %
% pdflatex -shell-escape main                                       %
% % % % % % % % % % % % % % % % % % % % % % % % % % % % % % % % % % % 


Krótka wstawka kodu w linii tekstu jest możliwa, np.  \lstinline|int a;| (biblioteka \texttt{listings})% lub  \mintinline{C++}|int a;| (biblioteka \texttt{minted})
. 
Dłuższe fragmenty lepiej jest umieszczać jako rysunek, np. kod na rys \ref{fig:pseudokod:listings}% i rys. \ref{fig:pseudokod:minted}
, a naprawdę długie fragmenty – w załączniku.


\begin{figure}
\centering
\begin{lstlisting}
class test : public basic
{
    public:
      test (int a);
      friend std::ostream operator<<(std::ostream & s, 
                                     const test & t);
    protected:
      int _a;  
      
};
\end{lstlisting}
\caption{Pseudokod w \texttt{listings}.}
\label{fig:pseudokod:listings}
\end{figure}

%\begin{figure}
%\centering
%\begin{minted}[linenos,frame=lines]{c++}
%class test : public basic
%{
%    public:
%      test (int a);
%      friend std::ostream operator<<(std::ostream & s, 
%                                     const test & t);
%    protected:
%      int _a;  
%      
%};
%\end{minted}
%\caption{Pseudokod w \texttt{minted}.}
%\label{fig:pseudokod:minted}
%\end{figure}




%--------------------------------------
%
%
% TODO: sposób testowania w ramach pracy, organizacja eksperymentów, przypadki testowe zakres testowania, wykryte i usunięte błędy, opcjonalnie wyniki badań eksperymentalnych
%
% Status: 0%
%--------------------------------------
\chapter{Weryfikacja i walidacja}
\label{ch:06}
\begin{itemize}
\item sposób testowania w ramach pracy (np. odniesienie do modelu V)
\item organizacja eksperymentów
\item przypadki testowe zakres testowania (pełny/niepełny)
\item wykryte i usunięte błędy
\item opcjonalnie wyniki badań eksperymentalnych
\end{itemize}

\begin{table}
\centering
\caption{Nagłówek tabeli jest nad tabelą.}
\label{id:tab:wyniki}
\begin{tabular}{rrrrrrrr}
\toprule
	         &                                     \multicolumn{7}{c}{metoda}                                      \\
	         \cmidrule{2-8}
	         &         &         &        \multicolumn{3}{c}{alg. 3}        & \multicolumn{2}{c}{alg. 4, $\gamma = 2$} \\
	         \cmidrule(r){4-6}\cmidrule(r){7-8}
	$\zeta$ &     alg. 1 &   alg. 2 & $\alpha= 1.5$ & $\alpha= 2$ & $\alpha= 3$ &   $\beta = 0.1$  &   $\beta = -0.1$ \\
\midrule
	       0 &  8.3250 & 1.45305 &       7.5791 &    14.8517 &    20.0028 & 1.16396 &                       1.1365 \\
	       5 &  0.6111 & 2.27126 &       6.9952 &    13.8560 &    18.6064 & 1.18659 &                       1.1630 \\
	      10 & 11.6126 & 2.69218 &       6.2520 &    12.5202 &    16.8278 & 1.23180 &                       1.2045 \\
	      15 &  0.5665 & 2.95046 &       5.7753 &    11.4588 &    15.4837 & 1.25131 &                       1.2614 \\
	      20 & 15.8728 & 3.07225 &       5.3071 &    10.3935 &    13.8738 & 1.25307 &                       1.2217 \\
	      25 &  0.9791 & 3.19034 &       5.4575 &     9.9533 &    13.0721 & 1.27104 &                       1.2640 \\
	      30 &  2.0228 & 3.27474 &       5.7461 &     9.7164 &    12.2637 & 1.33404 &                       1.3209 \\
	      35 & 13.4210 & 3.36086 &       6.6735 &    10.0442 &    12.0270 & 1.35385 &                       1.3059 \\
	      40 & 13.2226 & 3.36420 &       7.7248 &    10.4495 &    12.0379 & 1.34919 &                       1.2768 \\
	      45 & 12.8445 & 3.47436 &       8.5539 &    10.8552 &    12.2773 & 1.42303 &                       1.4362 \\
	      50 & 12.9245 & 3.58228 &       9.2702 &    11.2183 &    12.3990 & 1.40922 &                       1.3724 \\
\bottomrule
\end{tabular}
\end{table}  



%--------------------------------------
%
%
% TODO: uzyskane wyniki, kierunki ewentualnych dalszych prac, problemy napotkane w trakcie pracy, wnioski
%
% Status: 0%
%--------------------------------------
\chapter{Podsumowanie i wnioski}
\begin{itemize}
\item uzyskane wyniki w świetle postawionych celów i zdefiniowanych wyżej wymagań
\item kierunki ewentualnych danych prac (rozbudowa funkcjonalna …)
\item problemy napotkane w trakcie pracy
\end{itemize}



\backmatter

%\bibliographystyle{plplain}  % bibtex
%\bibliography{biblio} % bibtex
\printbibliography           % biblatex
\addcontentsline{toc}{chapter}{Bibliografia}

\begin{appendices}

%--------------------------------------
%
%
% TODO: skróty z tekstu
%
% Status: ?%
%--------------------------------------
\chapter{Spis skrótów i symboli}

\begin{itemize}
\item[DNA] kwas deoksyrybonukleinowy (ang. \english{deoxyribonucleic acid})
\item[MVC] model -- widok -- kontroler (ang. \english{model--view--controller}) 
\item[$N$] liczebność zbioru danych
\item[$\mu$] stopnień przyleżności do zbioru
\item[$\mathbb{E}$] zbiór krawędzi grafu
\item[$\mathcal{L}$] transformata Laplace'a 
\end{itemize}


%--------------------------------------
%
%
% TODO: długi fragment źródła
%
% Status: ?%
%--------------------------------------
\chapter{Źródła}

Jeżeli w pracy konieczne jest umieszczenie długich fragmentów kodu źródłowego, należy je przenieść w to miejsce.

\begin{lstlisting}
if (_nClusters < 1)
	throw std::string ("unknown number of clusters");
if (_nIterations < 1 and _epsilon < 0)
	throw std::string ("You should set a maximal number of iteration or minimal difference -- epsilon.");
if (_nIterations > 0 and _epsilon > 0)
	throw std::string ("Both number of iterations and minimal epsilon set -- you should set either number of iterations or minimal epsilon.");
\end{lstlisting}


% % % % % % % % % % % % % % % % % % % % % % % % % % % % % % % % % % % 
% Pakiet minted wymaga odkomentowania w pliku config/settings.tex   %
% importu pakietu minted: \usepackage{minted}                       %
% i specjalnego kompilowania:                                       %
% pdflatex -shell-escape praca                                      %
% % % % % % % % % % % % % % % % % % % % % % % % % % % % % % % % % % % 

%\begin{minted}[linenos,breaklines,frame=lines]{c++}
%if (_nClusters < 1)
%   throw std::string ("unknown number of clusters");
%if (_nIterations < 1 and _epsilon < 0)
%   throw std::string ("You should set a maximal number of iteration or minimal difference -- epsilon.");
%if (_nIterations > 0 and _epsilon > 0)
%   throw std::string ("Both number of iterations and minimal epsilon set -- you should set either number of iterations or minimal epsilon.");
%\end{minted}


%--------------------------------------
%
%
% TODO: źródła programu, dane testowe, film pokazujący działanie oprogramowania
%
% Status: 0%
%--------------------------------------
\chapter{Lista dodatkowych plików, uzupełniających tekst pracy} 


W systemie do pracy dołączono dodatkowe pliki zawierające:
\begin{itemize}
\item źródła programu,
\item dane testowe,
\item film pokazujący działanie opracowanego oprogramowania lub zaprojektowanego i~wykonanego urządzenia,
\item itp.
\end{itemize}


\listoffigures
\addcontentsline{toc}{chapter}{Spis rysunków}
\listoftables
\addcontentsline{toc}{chapter}{Spis tabel}

\end{appendices}

\end{document}


%% Finis coronat opus.

